%\documentclass[ignorenonframetext, compress, 9pt, xcolor=svgnames]{beamer} 
%\documentclass[ignorenonframetext, compress, 9pt, xcolor=svgnames]{beamer} 
\documentclass[notes, ignorenonframetext, compress, 10pt, xcolor=svgnames, aspectratio=169]{beamer} 
\input{../../../Preambule_Diapos}
\usepackage[svgnames]{xcolor}
\usepackage{tikz}
\usetikzlibrary{shapes.geometric, arrows}
\usepackage{enumerate}   
\usepackage{multirow}
\usepackage{txfonts}
\usepackage{mathrsfs}
\usepackage{pgfplots}
\pgfplotsset{compat = newest}
\usetikzlibrary{positioning, arrows.meta}
\usepgfplotslibrary{fillbetween}
\newcommand{\A}{(0,0) ++(135:2) circle (2)}
\newcommand{\B}{(0,0) ++(45:2) circle (2)}
\DeclareMathOperator{\C}{C}
\DeclareMathOperator{\util}{u}
%\setbeamersize{text margin left=1.5em,text margin right=1.5em} 
%\setbeamersize{text margin left=1.2cm,text margin right=1.2cm} 
\setbeamersize{text margin left=1.5em,text margin right=1.5em} 
%\usepackage{xr}
%\externaldocument{Econometrie1_UGA_P2e}
  \usepackage{eso-pic}
%\newcommand\AtPagemyUpperLeft[1]{\AtPageLowerLeft{%
%\put(\LenToUnit{0.9\paperwidth},\LenToUnit{0.85\paperheight}){#1}}}
%\AddToShipoutPictureFG{
 % \AtPagemyUpperLeft{{\includegraphics[width=1.1cm,keepaspectratio]{logoUGA2020.pdf}}}
%}%

%\setbeamercolor{title}{fg=black}
%\setbeamercolor{frametitle}{fg=black}
%\setbeamercolor{section in head/foot}{fg=black}
%\setbeamercolor{author in head/foot}{bg=Brown}
%\setbeamercolor{date in head/foot}{fg=Brown}
\AtBeginSection[]
  {
    \ifnum \value{framenumber}>1
      \begin{frame}<beamer>
      \frametitle{PLAN}
      \tableofcontents[currentsection]
      \end{frame}
    \else
    \fi
  }
\setbeamertemplate{section page}
{
    \begin{centering}
    \begin{beamercolorbox}[sep=11pt,center]{part title}
    \usebeamerfont{section title}\thesection.~\insertsection\par
    \end{beamercolorbox}
    \end{centering}
}
%\titlegraphic{\includegraphics[width=1cm]{logoUGA2020.pdf}}
\title[]{ 
    \textsc{Université de Grenoble Alpes\\(L3 Économie-Gestion, parcours E2AD)}\\
\textbf{ÉCONOMIE INDUSTRIELLE}\footnote{Responsable du cours: Alexis Garapin}}
\subtitle{TRAVAUX DIRIGÉS 8: \\ RELATIONS VERTICALES}
\date{\today}
\author{Michal W. Urdanivia\inst{*}}
\institute{\inst{*}UGA, Facult\'e d'\'Economie, GAEL, \\
e-mail:
 \href{
     mailto:michal.wong-urdanivia@univ-grenoble-alpes.fr}{michal.wong-urdanivia@univ-grenoble-alpes.fr}}

%\titlegraphic{\includegraphics[width=1cm]{logoUGA2020.pdf}
%}cludegraphics[width=1cm]{logoUGA2020.pdf}
%}

\begin{document}

%%% TIKZ STUFF
\usetikzlibrary{positioning}
\usetikzlibrary{snakes}
\usetikzlibrary{calc}
\usetikzlibrary{arrows}
\usetikzlibrary{decorations.markings}
\usetikzlibrary{shapes.misc}
\usetikzlibrary{matrix,shapes,arrows,fit,tikzmark}
\usetikzlibrary{matrix,chains,positioning,decorations.pathreplacing,arrows}
\usetikzlibrary{shapes}
\usetikzlibrary{shapes.geometric, arrows}
\tikzset{   
        every picture/.style={remember picture,baseline},
        every node/.style={anchor=base,align=center,outer sep=1.5pt},
        every path/.style={thick},
        }
\newcommand\marktopleft[1]{
    \tikz[overlay,remember picture] 
        \node (marker-#1-a) at (-.3em,.3em) {};%
}
\newcommand\markbottomright[2]{%
    \tikz[overlay,remember picture] 
        \node (marker-#1-b) at (0em,0em) {};%
}
\tikzstyle{every picture}+=[remember picture] 
\tikzstyle{mybox} =[draw=black, very thick, rectangle, inner sep=10pt, inner ysep=20pt]
\tikzstyle{fancytitle} =[draw=black,fill=red, text=white]
\tikzstyle{observed}=[draw,circle,fill=gray!50]

\begin{frame}
\titlepage
\end{frame}
\begin{frame}
 \tableofcontents
    \end{frame}
%\begin{frame}
%\frametitle{Contenu}
%\tableofcontents[pausesections, pausesubsections]
%\end{frame}

%\section{Qu'est-ce que l’économétrie ? A quoi (à qui) ça sert ?}
%\frame{\sectionpage}
%\begin{frame}
%  \tableofcontents  
%\end{frame}

\section{ ROLL et SKR}

\frame{\sectionpage}
\begin{frame}[allowframebreaks]{\insertsection}
\framesubtitle{Grille de correction}
\begin{enumerate}
\item \textbf{\underline{Rappel du contexte et des enjeux}}
\begin{enumerate}[$\cdot$]
    \item Une relation de moyen terme (depuis 5 ans), mais basée sur une succession de contrats de court terme
    \item Un fournisseur ROLL démotivé, pourtant pourvu de compétences spécifiques
    \item SKR constate cette démotivation et des problèmes de qualité, aimerait pourtant poursuivre sa relation d’affaires avec ROLL
    \item L’enjeu est de savoir s’il est possible de continuer cette relation, en identifiant ses causes et en proposant des solutions contractuelles plus adaptées
\end{enumerate}
\end{enumerate}
\end{frame}
%\begin{frame}[allowframebreaks]{References}
 %   \bibliographystyle{jpe}
  %  \bibliography{../../../Biblio}
   % \end{frame}
    
    \end{document}
    