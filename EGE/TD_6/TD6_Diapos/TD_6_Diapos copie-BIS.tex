%\documentclass[ignorenonframetext, compress, 9pt, xcolor=svgnames]{beamer} 
%\documentclass[ignorenonframetext, compress, 9pt, xcolor=svgnames]{beamer} 
\documentclass[notes, ignorenonframetext, compress, 10pt, xcolor=svgnames, aspectratio=169]{beamer} 
\input{../../../Preambule_Diapos}
\usepackage[svgnames]{xcolor}
\usepackage{tikz}
\usetikzlibrary{shapes.geometric, arrows}
\usepackage{enumerate}   
\usepackage{multirow}
\usepackage{txfonts}
\usepackage{mathrsfs}
\usepackage{pgfplots}
\pgfplotsset{compat = newest}
\usetikzlibrary{positioning, arrows.meta}
\usepgfplotslibrary{fillbetween}
\newcommand{\A}{(0,0) ++(135:2) circle (2)}
\newcommand{\B}{(0,0) ++(45:2) circle (2)}
\DeclareMathOperator{\C}{C}
\DeclareMathOperator{\util}{u}
%\setbeamersize{text margin left=1.5em,text margin right=1.5em} 
%\setbeamersize{text margin left=1.2cm,text margin right=1.2cm} 
\setbeamersize{text margin left=1.5em,text margin right=1.5em} 
%\usepackage{xr}
%\externaldocument{Econometrie1_UGA_P2e}
  \usepackage{eso-pic}
%\newcommand\AtPagemyUpperLeft[1]{\AtPageLowerLeft{%
%\put(\LenToUnit{0.9\paperwidth},\LenToUnit{0.85\paperheight}){#1}}}
%\AddToShipoutPictureFG{
 % \AtPagemyUpperLeft{{\includegraphics[width=1.1cm,keepaspectratio]{logoUGA2020.pdf}}}
%}%

%\setbeamercolor{title}{fg=black}
%\setbeamercolor{frametitle}{fg=black}
%\setbeamercolor{section in head/foot}{fg=black}
%\setbeamercolor{author in head/foot}{bg=Brown}
%\setbeamercolor{date in head/foot}{fg=Brown}
\AtBeginSection[]
  {
    \ifnum \value{framenumber}>1
      \begin{frame}<beamer>
      \frametitle{PLAN}
      \tableofcontents[currentsection]
      \end{frame}
    \else
    \fi
  }
\setbeamertemplate{section page}
{
    \begin{centering}
    \begin{beamercolorbox}[sep=11pt,center]{part title}
    \usebeamerfont{section title}\thesection.~\insertsection\par
    \end{beamercolorbox}
    \end{centering}
}
%\titlegraphic{\includegraphics[width=1cm]{logoUGA2020.pdf}}
\title[]{ 
    \textsc{Université de Grenoble Alpes\\(L3 Économie-Gestion, parcours EGE)}\\
\textbf{ÉCONOMIE INDUSTRIELLE}\footnote{Responsable du cours: Sylvain Rossiaud}}
\subtitle{TRAVAUX DIRIGÉS 6: \\ FUSIONS}
\date{\today}
\author{Michal W. Urdanivia\inst{*}}
\institute{\inst{*}UGA, Facult\'e d'\'Economie, GAEL, \\
e-mail:
 \href{
     mailto:michal.wong-urdanivia@univ-grenoble-alpes.fr}{michal.wong-urdanivia@univ-grenoble-alpes.fr}}

%\titlegraphic{\includegraphics[width=1cm]{logoUGA2020.pdf}
%}cludegraphics[width=1cm]{logoUGA2020.pdf}
%}

\begin{document}

%%% TIKZ STUFF
\usetikzlibrary{positioning}
\usetikzlibrary{snakes}
\usetikzlibrary{calc}
\usetikzlibrary{arrows}
\usetikzlibrary{decorations.markings}
\usetikzlibrary{shapes.misc}
\usetikzlibrary{matrix,shapes,arrows,fit,tikzmark}
\usetikzlibrary{matrix,chains,positioning,decorations.pathreplacing,arrows}
\usetikzlibrary{shapes}
\usetikzlibrary{shapes.geometric, arrows}
\tikzset{   
        every picture/.style={remember picture,baseline},
        every node/.style={anchor=base,align=center,outer sep=1.5pt},
        every path/.style={thick},
        }
\newcommand\marktopleft[1]{
    \tikz[overlay,remember picture] 
        \node (marker-#1-a) at (-.3em,.3em) {};%
}
\newcommand\markbottomright[2]{%
    \tikz[overlay,remember picture] 
        \node (marker-#1-b) at (0em,0em) {};%
}
\tikzstyle{every picture}+=[remember picture] 
\tikzstyle{mybox} =[draw=black, very thick, rectangle, inner sep=10pt, inner ysep=20pt]
\tikzstyle{fancytitle} =[draw=black,fill=red, text=white]
\tikzstyle{observed}=[draw,circle,fill=gray!50]

\begin{frame}
\titlepage
\end{frame}
\begin{frame}
 \tableofcontents
    \end{frame}
%\begin{frame}
%\frametitle{Contenu}
%\tableofcontents[pausesections, pausesubsections]
%\end{frame}

%\section{Qu'est-ce que l’économétrie ? A quoi (à qui) ça sert ?}
%\frame{\sectionpage}
%\begin{frame}
%  \tableofcontents  
%\end{frame}

\section{(Bref rappel) sur un modèle de Cournot de base à $N$ firmes}
\frame{\sectionpage}
\begin{frame}[allowframebreaks]{\insertsection}
\framesubtitle{Modèle}
\begin{itemize}
\item \textbf{\underline{Référence}}: \cite{belleflamme_peitz_2015}, chapitre 3.
\item $N$ firmes avec des fonctions de coûts symétriques à coûts marginaux constants:
\begin{align*}
    c_i(q_i) &=cq_i \Rightarrow c^m_i(q_i):= \frac{\partial c}{\partial q_i}(q_i) = c, \ c > 0, i=1, \ldots, N.
\end{align*}
\item La demande est représenté par une fonction demande inverse linéaire:
\begin{align*}
p(Q) &= a-bQ, \ a, b > 0, Q = \sum_{i=1}^N q_i
\end{align*}
\item Le profit d'une firme $i$ s'écrit alors: 
\begin{align*}
\pi_i(q_i, q_{(-i)}) = p(Q)q_i - c_i(q_i) =  (a-bQ)q_i - cq_i &= (a-b\sum_{i=1}^N q_i)q_i - cq_i\\
& = \left(a-b(q_i + \sum_{j=1, j\neq i}^N q_j)\right)q_i - cq_i\\
&= \left(a-b(q_i + q_{-i})\right)q_i - cq_i
\end{align*}
où on note $q_{-i}:= \sum_{j=1, j\neq i}^N q_j $ la quantité(totale) offerte par les autres firmes que la firme $i$.
\item Bien que $\pi_i$ dépende des quantités offertes par les autres firmes, 
la firme $i$ maximise $\pi_i$ par rapport à $q_i$ qui est sa seule variable de décision.
\item Autrement dit, chaque firme maximise son profit étant donné les décisions des autres.
\end{itemize}
\end{frame}

\begin{frame}[allowframebreaks]{\insertsection}
\framesubtitle{Fonctions de meilleure réponse}
    \begin{itemize}
        \item La c.p.o. devant être vérifiée par le choix optimal de la firme $i$, $q_i^*$ s'écrit: 
        \begin{align}
            \frac{\partial \pi_i}{\partial q_i}(q_i^*) =0 &\Leftrightarrow a-2bq_i -b\sum_{j=1, j\neq i }^N q_j - c = 0\\
            &\Leftrightarrow  a-2bq_i^* -b\left(Q - q_i^*\right) - c= 0,
            \label{eq1}
        \end{align}
         qui définit le choix optimal de la firme $i$ comme une fonction(implicite) de $q_{(-i)}$ qui apparaît 
         dans $Q$. 
         \item Cette fonction est la meilleure réponse de la firme $i$ qu'on note $q^{mr}_i(q_{-i})$ avec 
         $q_i^* = q^{mr}_i(q_{-i})$. Elle nous est donnée en utilisant \eqref{eq1} par: 
        \begin{align} 
            q_i^{mr}(q_{(-i)}) &= \frac{(a-c)}{b} - Q, \ i=1, \ldots, N.
            \label{eq2}
        \end{align}
           \item L'équilibre du marché est un équilibre de Nash du jeux en information complète, où
           la stratégie de chaque joueur est donné par \eqref{eq2}. 
           Autrement dit le vecteur des quantités d'équilibre $q_1^*, \ldots, q_N^*$ vérifie 
           \begin{align}
               q_i^*&= q_i^{mr}(q_{-i}^*), \ i=1, \ldots, N.
               \label{eq3}
           \end{align}
           où $q_i^{mr}$ est donnée par \eqref{eq2}. En notant $Q^*$ la quantité totale offerte 
           à l'équilibre \eqref{eq3} s'écrit:
           \begin{align*}
            q_i^*&=\frac{(a-c)}{b} - Q^*, \ i=1, \ldots, N.
           \end{align*}
           \item On peut utiliser cette dernière égalité et voir que $Q^*$ vérifie:
           \begin{align*}
            \sum_{i=1}^N  q_i^* = N\left(\frac{(a-c)}{b} - Q^*\right)
            &\Leftrightarrow Q^* = N\frac{(a-c)}{b} - NQ^*
            \Rightarrow  Q^* = \frac{N}{(N+1)}\frac{(a-c)}{b}.
           \end{align*}
           \item On obtient alors les quantités offertes par chaque firme à l'équilibre en utilisant
            \eqref{eq2}:
            \begin{align*} 
                q_i^{mr}(q_{-i^*}) &= \frac{(a-c)}{b} - Q^* = \frac{(a-c)}{(N+1)b}, \ i=1, \ldots, N.
            \end{align*}
            ce qui permets de calculer le prix et profits à l'équilibre: 
            \begin{align*}
                p^* &=p(Q^*) = a-bQ^* = a-b\left(\frac{N}{(N+1)}\frac{(a-c)}{b}\right)=\frac{a+Nc}{N+1},\\
                \pi_i^* &=\pi_i(q_i^*,q_{(-i)}^*)=p^*q_i^* - cq_i^* = (p^*-c)q_i^* =\frac{1}{b}\left(\frac{a-c}{N+1}\right)^2
            \end{align*}    
    \end{itemize}
\end{frame}

\begin{frame}[allowframebreaks]{\insertsection}
\framesubtitle{Fusion}
    \begin{itemize}
    \item \textbf{\underline{Référence}}: \cite{belleflamme_peitz_2015}, chapitre 15.
    \item Supposons que $K<N$  parmi les $N$ firmes fusionnent.
    \item Le nombre de firmes est alors:
    \begin{align*}
        \tilde{N} &= \underbrace{N-K}_{\substack{\text{firmes qui}\\\text{ne fusionnent pas}}} + \underbrace{1}_{\text{firme fusionnée}}
    \end{align*}
    \item Dans ce cas d'après les résultats précédents le profit de chaque firme est à l'équilibre:
    \begin{align*}
        \pi_{i, \tilde{N}}^* &= \frac{1}{b}\left(\frac{a-c}{\tilde{N}+1}\right)^2
        = \frac{1}{b}\left(\frac{a-c}{N-K+2}\right)^2.
    \end{align*}
    \item On se rappelle que leur profit avant la fusion est pour chacune:
    \begin{align*}
        \pi_i^* &= \frac{1}{b}\left(\frac{a-c}{N+1}\right)^2
    \end{align*}
    \item Ce qui donne un profit total pour les ayant fusionné avant la fusion de:
    \begin{align*}
        K\pi_i^* &= K\frac{1}{b}\left(\frac{a-c}{N+1}\right)^2
    \end{align*}
    \item Pour que la fusion soit profitable il faut alors que:
    \begin{align*}
        \pi_{i, \tilde{N}}^* \geq K\pi_i^*& \Leftrightarrow
         \frac{1}{b}\left(\frac{a-c}{N-K+2}\right)^2  \geq K\frac{1}{b}\left(\frac{a-c}{N+1}\right)^2\\
         & \Leftrightarrow (N+1)^2 \geq K(N-K+2)^2\\
    \end{align*}
    \item On peut alors considérer l'équation:
    \begin{align*}
        f(K) &= (N+1)^2 - K(N-K+2)^2,
    \end{align*}
    pour chercher les racines  de $f(K) = 0$, et ce qui permettra de définir le seuil $\tilde{K}$ 
    tel que pour $K > \tilde{K}$ $f(K) > 0$ et la fusion est profitable aux entreprises qui fusionnent.
    \item $f(K) = 0$ est une équation polynomiale de degré 3 dont les racines sont :
    \begin{align*}
        K_1, =1, K_2 = N - \sqrt{(4N + 5)}/2 + 3/2, K_3 = N + \sqrt{(4N + 5)}/2 + 3/2
    \end{align*}
    \item \textbf{Remarque}: 
    \begin{enumerate}[$\star$]
    \item il existe des méthodes connues pour résoudre ces équations, si vous n'en avez 
    vu commencez \href{https://www.lucaswillems.com/fr/articles/58/equations-troisieme-degre?cache=update}{ici par exemple}.
    \item ici, pour faire vite j'ai utilisé un outil de calcul symbolique avec dans Python avec 
    \href{https://www.sympy.org/en/index.html}{SimPy}.
    \end{enumerate}
    \item Parmi ces racines seulement la 2ème donne une région admissible donc $\tilde{K}=K_2$
    \item Si dans $\tilde{K}$ on fait varier $N$ on obtiendra qu'en général $\tilde{K}=0.8$.
    \item C'est ce qui est souvent appelé \textbf{règle du 80\%}: il faut que plus de 80\% des firmes fusionnent 
    pour que dans ce modèle cette fusion soit profitable à ces entreprises. Ici cela correspond à 9 firmes, 
    soit une situation de quasi monopole(en fait un duopole avec les firmes fusionnées et la firme restante)
\end{itemize}
\end{frame}

\section{Exercice d’application.}
\frame{\sectionpage}
\begin{frame}[allowframebreaks]{\insertsection}
    \framesubtitle{Avant fusion}
    \begin{itemize}
        \item Fonction de demande inverse: $p(Q) = 180-Q$, $Q=q_1+ \ldots + q_{9}$, 
        \item Fonction de coût: $c_i(q_i) = 30q_i$, pour $i=1, \ldots, 9$.
        \item $N = 9$ avant la fusion.
        \item Par conséquent:
        \begin{align*}
        Q^*_{(N=9)} &= \frac{N}{(N+1)}\frac{(a-c)}{b} = \frac{9}{(9+1)}\frac{(180-30)}{1} = 135,\\
        p^*_{(N=9)} &= \frac{a+Nc}{N+1} = \frac{180+9\times 30}{(9+1)} = 45,\\
        q_{(i, N=9)}^*&=  \frac{(a-c)}{(N+1)b} = \frac{(180-30)}{(9+1)\times 1} =15   ,\\
        \pi_{(i, N=9)}^*&= \frac{1}{1}\left(\frac{180-30}{9+1}\right)^2 = 225.
        \end{align*}
        \item Ce dernier chiffre donnant un surplus des producteurs de $SP_{(i, N=9)}^* = N\pi_{(N=9)}^*=9\times 225 = 2025.$
        \item  On calcule aussi le surplus du consommateur qui dans le cas d'une fonction de demande inverse 
        linéaire $p(Q) = a-bQ$ est donné par $SC(Q) = \frac{bQ^2}{2}$(voir \citet{belleflamme_peitz_2015}, chapitre 2).\\ Ici à l'équilibre 
        pour $Q^*_{(N=10)} = 135$,
         ce surplus est $SC_{(N=9)}^* = SC(Q_{(N=9)}^*) = \frac{135^2}{2}  =9112.5$.
        \end{itemize}
\end{frame}

\begin{frame}[allowframebreaks]{\insertsection}
    \framesubtitle{Après fusion}
    \begin{itemize}
        \item 2 firmes fusionnent et par conséquent $N=8$ désormais, d'où:
    \end{itemize}
        \begin{enumerate}[$\star$]
            \item $Q^*_{N=8} = 133,33$, 
            \item $p^*_{(N=8)}=46,67$, 
            \item $q_{(i, N=8)}^* = 16,67$,  
            \item $\pi_{(i, N=8)}^*= 277,89$. 
            \item Ce dernier chiffre donnant un surplus des producteurs de 
            $SP^* = N\pi_{(i, N=8)}^*=2528$.
            \item On calcule aussi le surplus du consommateur  
            $SC_{(N=8)}^* = SC(Q_{(N=8)}^*) = \frac{133,333^2}{2}$.
            \framebreak
            \item Ces résultats illustrent la règle du 80\% puisque le profit de la firme fusionnée 
            est inférieur au profit total des 2 firmes avant la fusion: $277,89 < 225\times 2$.
            \item Le profit des firmes ne participant pas à la fusion augmente. 
            Elles bénéficient indirectement de celle-ci.
            \item On note aussi que le surplus du consommateur diminue 
            avec la fusion tandis que celui des firmes augmente.
        \end{enumerate}
\end{frame}

        \begin{frame}[allowframebreaks]{\insertsection}
            \framesubtitle{Fusion et Stackelberg}
            \begin{itemize}
        \item Les 2 firmes qui fusionnent deviennent leader dans une concurrence du type Stackelberg. 
             \end{itemize}
        \begin{enumerate}[$\star$]
            \item Jeux séquentiel: le leader joue d'abord, les autres firmes suivent. 
            \item On utilise l'indice "$L$" pour le leader et $i$ pour les firmes qui suivent avec donc 
            ici $i=1, \ldots, 7$.
            \item La résolution  se fait par induction à rebours, c.à.d. selon la procédure suivante:
            \begin{enumerate}[(i)]
            \item d'abord le problème d'optimisation des firmes qui suivent étant donné le choix du leader(dernière étape du jeux),
            \item ensuite le problème d'optimisation du leader(première étape du jeux).
            \end{enumerate}
            \item Notons donc la quantité offerte par la firme leader(lc.à.d., issue des trois firmes qui ont fusionné) $q_L$. 
            \item \textbf{Dernière/2ème étape}: considérons une firme $i$ parmi les firmes qui n'ont pas fusionné, 
            et notons $q_i$ la quantité qu'elle offre. 
            \item La quantité totale offerte peut s'écrire:
            \begin{align*}
                Q&= \underbrace{q_i}_{\text{qté de $i$}} + \underbrace{q_L}_{\text{qté de $L$}} 
                + \underbrace{\sum_{j=1; j\neq i}^7 q_j}_{\substack{\text{qté des autres}\\ \text{firmes que $i$ et $L$}}}
                 =  q_i + q_L + \underbrace{q_{-j}}_{:=\sum_{j=1; j\neq i}^7 q_j}
            \end{align*}
            \item La demande peut s'écrire alors:
            \begin{align*}
                p(Q) &= 180 - ( q_{-i} + q_L + q_i ).
            \end{align*}
            \item Et le profit de la firme $i$ s'écrit:
            \begin{align*}
                \pi_i(q_i) &= \left[180 - (q_{-i} + q_L + q_i ) \right] - 30q_i = \left[150 - ( q_{-i} + q_L + q_i )\right]q_i.
            \end{align*}
            \item La c.p.o. associée à la maximisation du profit permet de définir la fonction
            de meilleure réponse de la firme $i$:
            \begin{align*}
             \frac{\partial \pi_i}{\partial q_i}(q_i^{*}) = 0 &\Rightarrow 
             q_i^{*} = q_i^{mr}(q_{-i}, q_L) = 75 - \frac{(q_{-i} + q_L)}{2}.
             \end{align*}
             \begin{enumerate}[$\star$]
                \item Notons $q_{-i}^{*}$ l'analogue à l'équilibre de $q_{-i}$, et $q_L^*$ la quantité offeret
               à l'équilibre par la firme leader(le trois ayant fusionné). 
               \item Pour simplifier utilisons le fait que les firmes qui suivent 
               sont supposées identiques(mêmes fonction de coût) et en conséquence 
               à l'équilibre elles sont caractérisés par les mêmes valeurs quant aux quantités offertes, profits, etc.  
               \item Par conséquent $q_{-i}^{*} = \sum_{j=1; j\neq i}^7 q_i^* = 6q_i^*$.
              \end{enumerate}
              \item En utilisant la meilleure réponse de $i$ on obtient alors:
              \begin{align*}
                  q_i^{*}= 75 - \frac{(6q_i^* + q_L^*)}{2} &\Rightarrow q_i^* = 18.75 -\frac{q_L^*}{8}.
                  \end{align*}

             \framebreak
             \item \textbf{1ère étape}:
             \item On considère le profit de la firme leader qui connaît les meilleures réponses des firmes 
             qui suivent obtenues à l'étape précédente.  
             \item Le profit de la firme leader s'écrit:
             \begin{align*}
                \pi_L(q_L) &= P(Q)-cq_L =  \left(180 -q_L - \sum_{i=1}^7 q_i\right)q_L - 30q_L 
                = \left(150 -q_L + \sum_{i=1}^7 q_i\right) q_L
             \end{align*}
             \item En particulier, à l'optimum ce profit peut s'écrire:
             \begin{enumerate}[$\star$]
                \item Peut s'écrire:
                \begin{align*}
                \pi_L(q_L^*) = \left(150 -q_L^* - \sum_{i=1}^7 \underbrace{q_i^*}_{18.75 -\frac{q_L^*}{8}}\right)q_L^* 
                &= \left(150 -q_L^* - 7\left(18.75 -\frac{q_L^*}{8}\right)\right)q_L^*\\
                &= \left(18.75-\frac{q_L^*}{8}\right)q_L^*\\
                &= 18.75q_L^* -\frac{q_L{^{*^2}}}{8}
               \end{align*}
               \item Et par définition de $q_L^*$ comme maximisant $\pi_L(\cdot)$, il doit vérifier(c.p.o.):  
               \begin{align*}
                \frac{\partial \pi_L}{\partial q_L}(q_L^*) = 0 &\Rightarrow q_L^* = 75,
               \end{align*}
               et on en déduit alors: 
               
               \item la quantité offerte par la firme $i$ à l'équilibre: 
               \begin{align*}
                q_i^* &= q_i^* = 18.75 -\frac{q_L^*}{8} =9.375, \ \text{pour} \ i=1, \ldots, 7.
               \end{align*}
               \item La quantité totale offerte et le prix d'équilibre:
               \begin{align*}
                Q^* = 7\times q_i^* + q_L^* = 140.625 &\Rightarrow p^* = P(Q^*) = 180-Q^* = 39.75.
               \end{align*}
               \item Les profits des firmes à l'équilibre:
               \begin{align*}
               \pi_L^* = (p^* - 40)q_L^* = 731.25&, \ \pi_i^* = (p^* - 40)q_i^* = 91.4.
               \end{align*}
               \item  Les surplus des agents sur le marché(c.à.d., firmes et consommateurs):
               \begin{align*}
                   SP^* &= \underbrace{7\times \pi_i^* + \pi_L^*}_{\text{surplus des firmes}} ,\\ 
                   SC^* &=  \underbrace{\frac{Q^{*^2}}{2}}_{\substack{\text{surplus des }\\
                   \text{consommateurs}}}.
               \end{align*}
            
             \end{enumerate}
            \end{enumerate}
            
\end{frame}

    \begin{frame}[allowframebreaks]{Commentaires/Résumé}
    \begin{itemize}
        \item La fusion est considérée dans le cadre d'un modèle de Cournot avec des firmes
         symétriques quant à leurs caractéristiques(mêmes coûts), et une demande linéaire. 
        \item Le premier point de l'exercice est simplement une extension du duopole 
        de Cournot classique à $N>2$ firmes(ici $N = 10$) qui en est un cas particulier. 
        \item Le deuxième point illustre les conditions restrictives 
        d'un fusion profitable(aux firmes) dans ce type de modèle qui est résumé par la \textbf{règle du 80\%}. 
        \item Le dernier point introduit la possibilités de "synergies" entre 
        entreprises fusionnées laquelle se traduirait par l'acquisition d'un 
        statut de leader de la firme issue de la fusion(remarque: 
        sans qu'il ne soit dit d'où proviennent ces synergies)
    \end{itemize}
    \end{frame}
\begin{frame}[allowframebreaks]{References}
    \bibliographystyle{jpe}
    \bibliography{../../../Biblio}
    \end{frame}
    
    \end{document}
    