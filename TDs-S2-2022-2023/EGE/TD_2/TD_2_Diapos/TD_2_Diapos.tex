%\documentclass[ignorenonframetext, compress, 9pt, xcolor=svgnames]{beamer} 
\input{../../../Config_diapos}
\usepackage[svgnames]{xcolor}
\usepackage{tikz}
\usetikzlibrary{shapes.geometric, arrows}
\usepackage{enumerate}   
\usepackage{multirow}
\usepackage{txfonts}
\usepackage{mathrsfs}
\usepackage{pgfplots}
\pgfplotsset{compat = newest}
\usetikzlibrary{positioning, arrows.meta}
\usepgfplotslibrary{fillbetween}
\newcommand{\A}{(0,0) ++(135:2) circle (2)}
\newcommand{\B}{(0,0) ++(45:2) circle (2)}
\DeclareMathOperator{\C}{C}
\DeclareMathOperator{\util}{u}
%\setbeamersize{text margin left=1.5em,text margin right=1.5em} 
%\setbeamersize{text margin left=1.2cm,text margin right=1.2cm} 
\setbeamersize{text margin left=1.5em,text margin right=1.5em} 
%\usepackage{xr}
%\externaldocument{Econometrie1_UGA_P2e}
  \usepackage{eso-pic}
%\newcommand\AtPagemyUpperLeft[1]{\AtPageLowerLeft{%
%\put(\LenToUnit{0.9\paperwidth},\LenToUnit{0.85\paperheight}){#1}}}
%\AddToShipoutPictureFG{
 % \AtPagemyUpperLeft{{\includegraphics[width=1.1cm,keepaspectratio]{logoUGA2020.pdf}}}
%}%

%\setbeamercolor{title}{fg=black}
%\setbeamercolor{frametitle}{fg=black}
%\setbeamercolor{section in head/foot}{fg=black}
%\setbeamercolor{author in head/foot}{bg=Brown}
%\setbeamercolor{date in head/foot}{fg=Brown}
\AtBeginSection[]
  {
    \ifnum \value{framenumber}>1
      \begin{frame}<beamer>
      \frametitle{PLAN}
      \tableofcontents[currentsection]
      \end{frame}
    \else
    \fi
  }
\setbeamertemplate{section page}
{
    \begin{centering}
    \begin{beamercolorbox}[sep=11pt,center]{part title}
    \usebeamerfont{section title}\thesection.~\insertsection\par
    \end{beamercolorbox}
    \end{centering}
}
%\titlegraphic{\includegraphics[width=1cm]{logoUGA2020.pdf}}
\title[]{ \textbf{ÉCONOMIE INDUSTRIELLE}\footnote{Responsable du cours: Sylvain Rossiaud}\\(\textbf{UGA, L3 EGE, S2})}
\subtitle{TRAVAUX DIRIGÉS: TD 2\\ LES BARRIÈRES STRATÉGIQUES À L'ENTRÉE }
\date{\today}
\author{Michal W. Urdanivia\inst{*}}
\institute{\inst{*}UGA, Facult\'e d'\'Economie, GAEL, \\
e-mail:
 \href{
     mailto:michal.wong-urdanivia@univ-grenoble-alpes.fr}{michal.wong-urdanivia@univ-grenoble-alpes.fr}}

%\titlegraphic{\includegraphics[width=1cm]{logoUGA2020.pdf}
%}

\begin{document}

%%% TIKZ STUFF
\usetikzlibrary{positioning}
\usetikzlibrary{snakes}
\usetikzlibrary{calc}
\usetikzlibrary{arrows}
\usetikzlibrary{decorations.markings}
\usetikzlibrary{shapes.misc}
\usetikzlibrary{matrix,shapes,arrows,fit,tikzmark}
\usetikzlibrary{matrix,chains,positioning,decorations.pathreplacing,arrows}
\usetikzlibrary{shapes}
\usetikzlibrary{shapes.geometric, arrows}
\tikzset{   
        every picture/.style={remember picture,baseline},
        every node/.style={anchor=base,align=center,outer sep=1.5pt},
        every path/.style={thick},
        }
\newcommand\marktopleft[1]{
    \tikz[overlay,remember picture] 
        \node (marker-#1-a) at (-.3em,.3em) {};%
}
\newcommand\markbottomright[2]{%
    \tikz[overlay,remember picture] 
        \node (marker-#1-b) at (0em,0em) {};%
}
\tikzstyle{every picture}+=[remember picture] 
\tikzstyle{mybox} =[draw=black, very thick, rectangle, inner sep=10pt, inner ysep=20pt]
\tikzstyle{fancytitle} =[draw=black,fill=red, text=white]
\tikzstyle{observed}=[draw,circle,fill=gray!50]

\begin{frame}
\titlepage
\end{frame}
\begin{frame}
 \tableofcontents
    \end{frame}
%\begin{frame}
%\frametitle{Contenu}
%\tableofcontents[pausesections, pausesubsections]
%\end{frame}

%\section{Qu'est-ce que l’économétrie ? A quoi (à qui) ça sert ?}
%\frame{\sectionpage}
%\begin{frame}
%  \tableofcontents  
%\end{frame}

\section{Exercice 1}
\frame{\sectionpage}
\begin{frame}[allowframebreaks]{\insertsection}
\framesubtitle{(a) Demande résiduelle}
\begin{itemize}
    \item Deux firmes aux coûts respectifs: 
    \begin{align}
     c_1(q_1) &= 40q_1 \ (\text{firme 1}) \\
     c_2(q_2) &= 
     \underbrace{100}_{\substack{\text{part fixe}\\\text{non} \\ \text{récupérable}}} + 40q_2 \ \text{(firme 2)}
     \label{eq18}
    \end{align}
    \item \textbf{\underline{Remarque}}: la part fixe dans le coût de 2 donne un avantage 
    à 1 qui est en place par rapport à 2 qui est l'entrant potentiel.
    \item La demande sur le marché est donné par: 
    \begin{align}
        P&= P(Q) = 100 - \underbrace{(q_1 + q_2)}_{= Q}
        \label{eq19}
    \end{align}
    \item \textbf{\underline{Remarque}}: on change un peu les notations par rapport à l'énoncé où $q_1=Q$, et $q_2 = q$. Ici donc  
    $Q$ est la quantité totale et $q_i$ celle produite par la firme $i=1, 2$.
    \item \eqref{eq19} permet d'avoir la demande(inverse) résiduelle qui s'adresse à 2 lorsque 1 produit par exemple $q_1=Q_0\geq 0$: $P=100-Q_0-q_2$.
\end{itemize}
\end{frame}

\begin{frame}[allowframebreaks]{\insertsection}
\framesubtitle{(b) choix de 2}
    \begin{itemize}
    \item \textbf{\underline{Réponse de 2 pour $q_1$ donné}}:  
    \begin{enumerate}[-]
        \item Profit de 2:  
        \begin{align}
            \pi_2(q_2) &= \underbrace{Pq_2}_{=R_2(q_2)(\text{Recette})} -c_2(q_2) 
            = \left(\underbrace{100 - (q_1 + q_2)}_{=P(\text{par \eqref{eq19}})}\right)q_2 -
             \left( \underbrace{100 + 40q_2}_{=c_2(q_2)(\text{par \eqref{eq18}})}\right).
             \label{eq20}
        \end{align}
        \item Notons $q_2^*$ le quantité qui maximise \eqref{eq20}, 
        et peut être définie à partir de la c.p.o.,
        \begin{align}
            \frac{\partial \pi_2}{\partial q_2}(q_2^*) &=0\Leftrightarrow 60 - q_1-2q_2^{*}=0\Rightarrow 
            q_2^{*} = 30 - \frac{q_1}{2}\eqqcolon q_2^{mr}(q_1).
            \label{eq21}
        \end{align}
        \item En particulier pour $q_1=Q_0$, le choix optimal de 2 sera $q_2^* = 30 - \frac{Q_0}{2}$.
    \end{enumerate}
\end{itemize}
\end{frame}

\begin{frame}[allowframebreaks]{\insertsection}
\framesubtitle{c) Stratégie de "prix limite" de 1}
    \begin{itemize}
        \item Pour 1 la stratégie consiste à choisir de produire une quantité $q_1^L$ telle qu'elle dissuade 2 
        de décider d'entrer. 
        \item Ce sera le cas(puisque les agents maximisent leur profit) si le profit de 2 est nul pour $q_1^L$.
        \item On a d'après \eqref{eq20}:
        \begin{align}
            \pi_2(q_2) &= (100 - (q_1 + q_2))q_2 - (100 + 40q_2) = (60 - q_1-q_2)q_2 - 100.
         \label{eq22}
        \end{align}
        \item D'après \eqref{eq22}:
        \begin{align}
            \pi_2(q_2) &= 0 \Leftrightarrow  (60 - q_1-q_2)q_2 - 100 = 0.
            \label{eq23}
        \end{align}
        \item Lorsque 1 choisit $q_1^L$ telle que \eqref{eq23} le niveau 
        que 2 décide est donné par $q_2^{mr}(q_1^L)$ en \eqref{eq21}. D'où:
        \begin{align*}
            \left(60-q_1^L - \left(\underbrace{30 - \frac{q_1^L}{2}}_{=q_2=:q_2^{mr}(q_1^L)}\right)\right)\left(
                \underbrace{30 - \frac{q_1^L}{2}}_{=q_2=:q_2^{mr}(q_1^L)}\right) - 100 = 0& \Leftrightarrow 
                \left(30-\frac{q_1^L}{2}\right)^2 -100=0,
        \end{align*}
        \item Qui est une équation polynomiale de degré 2 avec deux racine $q_1^L = 40$ et $q_1^L = 80$.
        \item Considérons $q_1^L = 40$. Dans ce cas $q_2^L = 10$, $Q^L = 50$, et $P^L = 50$, $\pi_1(q_1^L) = 400$.
        \item Lorsque 1 choisit $q_1^L$ elle ne maximise pas son profit 
        mais en dissuadant 2 d'entrer(car son profit est nul) elle peut 
        s'attendre à bénéficier de sa situation de monopole par la suite:
        \begin{enumerate}[-]
            \item En monopole 1 maximise
            \begin{align*}
                \pi_1(q_1) &= \underbrace{P(q_1)q_1}_{=:R_1(q_1)(\text{recette})} -c_1(q_1) = (100-q_1)q_1 - 40q_1.
            \end{align*}
            \item Notons $q_1^M$ le choix optimale de monople qui vérifie(c.p.o.): 
            \begin{align*}
                \frac{\partial \pi(q_1^M)}{\partial q_1} &= 0 \Leftrightarrow 60-2q_1^M = 0\Rightarrow q_1^M = 30,
            \end{align*}
            et alors $P^M=P(q_1^M) = 70$, et $\pi_1(q_1^M) = 900 > \pi_1(q_1^L)$.
            \item Pour le choix $q_1^L = 80$(deuxième racine), le profit de 1 est négatif de sorte ce cela n'est pas un choix optimal.
        \end{enumerate}
    \end{itemize}
    \end{frame}
    
\section{Exercice 2}
\frame{\sectionpage}
\begin{frame}[allowframebreaks]{\insertsection}
\framesubtitle{Modèle}
\begin{itemize}
\item Une firme(la firme 1) produit en monopole un produit en quantité $q_1$.
\item Le demande est donnée par la fonction de demande inverse: 
\begin{align*}
    P(Q) &= 50 - \frac{Q}{10}, 
\end{align*}
où $Q = q_1$ dès lors que la firme est en monopole.
\item Et ce faisant sa recette peut s'écrire,
\begin{align}
    R_1(q_1) &= P(q_1)q_1 = 50q_1- \frac{q_1^2}{10} \Rightarrow R^m_1(q_1) := \frac{\partial R_1(q_1)}{\partial q_1} = 50-\frac{q_1}{5}
\label{eq8}
\end{align}
où $R^m_1(q_1)$ est la recette marginale.
\item Son coût est supposé:
\begin{align}
    c_1(q_1) &= \frac{q_1^2}{40} \Rightarrow c^m_1(q_1) = \frac{q_1}{20} > 0, \ \text{pour tout \ $q_1>0$(coût marginal croissant)}
\label{eq9}
\end{align}
\end{itemize}
    \end{frame}

    \begin{frame}[allowframebreaks]{\insertsection}
\framesubtitle{(a) Équilibre de monopole}
        \begin{itemize}
            \item La firme maximise son profit qui ne dépend que de $q_1$:
            \item Profit de 1: 
            \begin{align}
                \pi_1(q_1) &=R_1(q_1) - c_1(q_1),
                \label{eq10}
            \end{align}
            et il est facile de voir que la quantité $q_1^*$ qui maximise \eqref{eq10} vérifie(c.p.o.)
            \begin{align}
                \frac{\partial \pi_1(q_1^*)}{\partial q_1} &= 0 \Leftrightarrow  R^m_1(q_1^*) = c^m_1(q_1^*),
                \label{eq11}
            \end{align}
            d'où par \eqref{eq8}, \eqref{eq9} et \eqref{eq10}:
            \begin{align*}
                 50-\frac{q_1^*}{5}=\frac{q_1^*}{20} \Rightarrow q_1^* = 200,
            \end{align*}
            le prix d'équilibre étant $P^* := P(q_1^*) = 30$.
        \end{itemize}
    \end{frame}   


\begin{frame}[allowframebreaks]{\insertsection}
\framesubtitle{(b) Marché contestable(2ème firme)}
    \begin{itemize}
        \item Produire est plus coûteux pour 2 avec: 
        \begin{align}
            c_2(q_2) &= 10q_2 + \frac{q_2^2}{40} \Rightarrow c^m_2(q_2) = 10 + \frac{q_2}{20}.
            \label{eq12}
        \end{align}
        \item  \textbf{\underline{Demande résiduelle}} pour 2 quand 1 conserve le niveau de production $q_1= q_1^* = 200$. 
        \begin{enumerate}[-]
            \item La recette de 2 comme fonction de $q_1$ est:
            \begin{align}
                R_2(q_2) &=Pq_2= P(\underbrace{q_1^* + q_2}_{=Q})q_2 = 
                \left(50 - \frac{(q_1^* + q_2)}{10}\right)q_2 = \frac{(500-q_1)q_2}{10} -\frac{q_2^2}{10}
                \label{eq13}
            \end{align}
            \item Et son profit peut s'écrit:  
            \begin{align}
                \pi_2(q_2) &= R_2(q_2) - c_2(q_2) = \underbrace{\frac{(500-q_1)q_2}{10} -\frac{q_2^2}{10}}_{=R_2(q_2) \ \text{par \eqref{eq13}}}
                - \underbrace{(10q_2 + \frac{q_2^2}{40})}_{c_2(q_2),  \ \text{par \eqref{eq12}}}
                \label{eq14}
            \end{align}
            \item On note $q_2^*$ la quantité qui maximise \eqref{eq14} et vérifie donc(c.p.o.):  
            \begin{align}
                \frac{\partial \pi(q_2^*)}{\partial q_2} &=0 \Leftrightarrow \frac{(500-q_1)}{10} - 10 -  \frac{q_2^*}{4} = 0
                 \Rightarrow q_2^* = 160 - \frac{2q_1}{5}=: q_2^{mr}(q_1),
                 \label{eq15}
            \end{align}
            et pour $q_1 = q_1^*=200$, on obtient $q_2^* = 80$ d'où  $Q^* = q_1^* + q_2^* = 280$, $P^* := P(Q^*) = 22$.
        \end{enumerate}
    \end{itemize}
\end{frame}   

\begin{frame}[allowframebreaks]{\insertsection}
\framesubtitle{(c) Stratégie de prix limite}
\begin{itemize}
\item 2 n'a que des coût variables et ce faisant $\pi_2 = 0 \Leftrightarrow q_2 = 0$.  
\item En utilisant  $q_2^{mr}(q_1)$ définie dans \eqref{eq15} on a: 
\begin{align*}
    q_2^{mr}(q_1) &= 0 \Leftrightarrow 160 - \frac{2q_1}{5} = 0 \Leftrightarrow q_{(1, q_2=0)} = 400=:q_1^L,
\end{align*}
où l'indice inf "$q_2=0$" est là pour indiquer que c'est le niveau de $q_1$ tel que $q_2=0$,  et 
qui définit ici le \textbf{\underline{la quantité associée au prix limite}}, c.à.d., pour lequel $\pi_2=0$, qu'on note $q_1^L$ ci-dessus. 
\item On note $Q^L = q_1^L + 0$ la quantité totale produite, et $P^L:=P(Q^L) = 10$, le prix limite.
\item On calcule aussi le profit obtenu par 1: 
\begin{align*}
    \pi_1(q_1^L)&=P^L q_1^L - c_1(q_1^L) = 0
\end{align*}
\end{itemize}
\end{frame}

\begin{frame}[allowframebreaks]{\insertsection}
\framesubtitle{(d) Cournot}
    \begin{itemize}
        \item En concurrence à la Cournot le profit de 1 s'écrit: 
        \begin{align}
            \pi_1(q_1, q_2) &= P(\underbrace{Q}_{=q_1+ q_2})q_1 - c_1(q_1) = \left(50-\frac{(q_1+q_2)}{10}\right)q_1  - \frac{q_1^2}{40}   -
            =  50 q_1 -\frac{q_1q_2}{10} - \frac{q_1^2}{8},
            \label{eq16}
        \end{align}
        qu'on maximise pour obtenir $q_1^{*c}$ la quantité qui maximise le profit de 1 en Cournot. Elle vérifie(c.p.o.)
        \begin{align}
            \frac{\partial \pi_1(q_1^{*c}, q_2)}{\partial q_1} &= 0 \Leftrightarrow 50 - \frac{q_2}{10} -  
            \frac{q_1^{*c}}{4} = 0 \Rightarrow q_1^{*c} = 200-\frac{2 q_2}{5} =: q_1^{mr}(q_2).
            \label{eq17}
        \end{align}
        \item En utilise les meilleures réponses $q_1^{mr}(q_2)$ donnée ci-dessus et $q_2^{mr}(q_1)$ donnée en \eqref{eq15} pour obtenir: 
        \begin{align*}
            q_1^{*c} &= 200-\frac{2}{5}\left( \underbrace{160 - \frac{2q_1^{*c}}{5}}_{=q_2^{mr}} \right)   
            \Leftrightarrow \frac{21q_1^{*c}}{25} = 136     \Rightarrow    q_1^{*c} \approx 161.9
         \end{align*}
         d'où, 
         \begin{align*}
            q_2^{*c} &= q_2^{mr}(q_1^{*c}) \approx 95.2.
        \end{align*}
        \item On calcule aussi $Q^{*c} = q_1^{*c} + q_2^{*c}\approx 257.1$ et $P^{*c} = P(Q^{*c}) \approx 24.3$.
        \item Finalement comme: 
        \begin{align*}
          \pi_1(q_1^{*c}) &=  P^{*c}q_1^{*c} - c_1(q_1^{*c}) \approx 3278.9 > \pi_1(q_1^L) = 0
        \end{align*}
        la stratégie de prix limite par 1 n'apparaît pas crédible du point de vue de 2, car 1 a intérêt à une
         concurrence à la Cournot qui lui permet un profit non nul. 
    \end{itemize}
\end{frame}
%\begin{frame}[allowframebreaks]{Références}
%\bibliographystyle{jpe}
%\bibliography{../../../Biblio}
%\end{frame}

\end{document}
