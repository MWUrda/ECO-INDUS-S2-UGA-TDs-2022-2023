%\documentclass[ignorenonframetext, compress, 9pt, xcolor=svgnames]{beamer} 
\input{../../../Config_diapos}
\usepackage[svgnames]{xcolor}
\usepackage{tikz}
\usetikzlibrary{shapes.geometric, arrows}
\usepackage{enumerate}   
\usepackage{multirow}
\usepackage{txfonts}
\usepackage{mathrsfs}
\usepackage{pgfplots}
\pgfplotsset{compat = newest}
\usetikzlibrary{positioning, arrows.meta}
\usepgfplotslibrary{fillbetween}
\newcommand{\A}{(0,0) ++(135:2) circle (2)}
\newcommand{\B}{(0,0) ++(45:2) circle (2)}
\DeclareMathOperator{\C}{C}
\DeclareMathOperator{\util}{u}
%\setbeamersize{text margin left=1.5em,text margin right=1.5em} 
%\setbeamersize{text margin left=1.2cm,text margin right=1.2cm} 
\setbeamersize{text margin left=1.5em,text margin right=1.5em} 
%\usepackage{xr}
%\externaldocument{Econometrie1_UGA_P2e}
  \usepackage{eso-pic}
%\newcommand\AtPagemyUpperLeft[1]{\AtPageLowerLeft{%
%\put(\LenToUnit{0.9\paperwidth},\LenToUnit{0.85\paperheight}){#1}}}
%\AddToShipoutPictureFG{
 % \AtPagemyUpperLeft{{\includegraphics[width=1.1cm,keepaspectratio]{logoUGA2020.pdf}}}
%}%

%\setbeamercolor{title}{fg=black}
%\setbeamercolor{frametitle}{fg=black}
%\setbeamercolor{section in head/foot}{fg=black}
%\setbeamercolor{author in head/foot}{bg=Brown}
%\setbeamercolor{date in head/foot}{fg=Brown}
\AtBeginSection[]
  {
    \ifnum \value{framenumber}>1
      \begin{frame}<beamer>
      \frametitle{PLAN}
      \tableofcontents[currentsection]
      \end{frame}
    \else
    \fi
  }
\setbeamertemplate{section page}
{
    \begin{centering}
    \begin{beamercolorbox}[sep=11pt,center]{part title}
    \usebeamerfont{section title}\thesection.~\insertsection\par
    \end{beamercolorbox}
    \end{centering}
}
%\titlegraphic{\includegraphics[width=1cm]{logoUGA2020.pdf}}
\title[]{ \textbf{ÉCONOMIE INDUSTRIELLE}\footnote{Responsable du cours: Sylvain Rossiaud}\\(\textbf{UGA, L3 EGE, S2})}
\subtitle{TRAVAUX DIRIGÉS: TD 4\\ ENTENTES VERTICALES }
\date{\today}
\author{Michal W. Urdanivia\inst{*}}
\institute{\inst{*}UGA, Facult\'e d'\'Economie, GAEL, \\
e-mail:
 \href{
     mailto:michal.wong-urdanivia@univ-grenoble-alpes.fr}{michal.wong-urdanivia@univ-grenoble-alpes.fr}}

%\titlegraphic{\includegraphics[width=1cm]{logoUGA2020.pdf}
%}cludegraphics[width=1cm]{logoUGA2020.pdf}
%}

\begin{document}

%%% TIKZ STUFF
\usetikzlibrary{positioning}
\usetikzlibrary{snakes}
\usetikzlibrary{calc}
\usetikzlibrary{arrows}
\usetikzlibrary{decorations.markings}
\usetikzlibrary{shapes.misc}
\usetikzlibrary{matrix,shapes,arrows,fit,tikzmark}
\usetikzlibrary{matrix,chains,positioning,decorations.pathreplacing,arrows}
\usetikzlibrary{shapes}
\usetikzlibrary{shapes.geometric, arrows}
\tikzset{   
        every picture/.style={remember picture,baseline},
        every node/.style={anchor=base,align=center,outer sep=1.5pt},
        every path/.style={thick},
        }
\newcommand\marktopleft[1]{
    \tikz[overlay,remember picture] 
        \node (marker-#1-a) at (-.3em,.3em) {};%
}
\newcommand\markbottomright[2]{%
    \tikz[overlay,remember picture] 
        \node (marker-#1-b) at (0em,0em) {};%
}
\tikzstyle{every picture}+=[remember picture] 
\tikzstyle{mybox} =[draw=black, very thick, rectangle, inner sep=10pt, inner ysep=20pt]
\tikzstyle{fancytitle} =[draw=black,fill=red, text=white]
\tikzstyle{observed}=[draw,circle,fill=gray!50]

\begin{frame}
\titlepage
\end{frame}
\begin{frame}
 \tableofcontents
    \end{frame}
%\begin{frame}
%\frametitle{Contenu}
%\tableofcontents[pausesections, pausesubsections]
%\end{frame}

%\section{Qu'est-ce que l’économétrie ? A quoi (à qui) ça sert ?}
%\frame{\sectionpage}
%\begin{frame}
%  \tableofcontents  
%\end{frame}

\section{Rappels sur la double marge et exercice d'application}
\frame{\sectionpage}
\begin{frame}[allowframebreaks]{Le cadre}
\begin{itemize}
\item \textbf{\underline{Référence}}: \cite{belleflamme_peitz_2015}, chapitre 17(section 17.1).
\item Cadre: des firmes dans une chaîne de production/offre verticale. 
\item Exemple:
\begin{enumerate}[$\star$]
    \item Une firme produit un bien intermédiaire(c.à.d, l'input) dans la production d'un bien final 
    par une autre firme. 
    \item Une firme produit un bien final qui est distribué sur le marché par une autre firme. 
    \item \ldots
\end{enumerate} 
\item Chaque firme associée à un niveau de la chaîne possède potentiellement un pouvoir de marché 
et la modélisation doit en tenir compte pour pouvoir décrire les marchés considérés et répondre à des questions telles que: 
\begin{enumerate}[$\star$]
\item L'effet de l'intégration verticale?
\item Une firme en amont(e.g., productrice d'un bien) peut-elle empêcher des concurrents 
d'accéder en faisant des accords d'exclusivité avec des firmes en aval(e.g., les distributeurs)?
\end{enumerate} 

\framebreak

\item L'inefficacité du fonctionnement de ces marchés est le problème de la  \textbf{\underline{Double marge}}:
\begin{enumerate}[$\star$]
\item Sur un marché dans lequel les firmes n'opèrent qu'à un seul niveau de la chaîne verticale d'offre 
les prix de distribution sont plus élevés que sur un marché verticalement intégré car une firme 
en aval(c.à.d., le distributeur) applique une marge sur le prix de vente du bien final qui intègre la marge appliquée 
par la firme en amont(c.à.d., le producteur).
\end{enumerate}
\end{itemize}
\end{frame}
\begin{frame}[allowframebreaks]{Illustration be base: prix linéaires et double marge}
\begin{itemize}
\item Marché monopolistique pour un bien final. 
\item Deux firmes à deux niveaux de la chaîne: une firme en amont appelée "producteur", firme en aval appelée "distributeur".
\item Demande sur le marché: 
\begin{align*}
    q(p) &= a-bp, \quad a, b > 0
\end{align*}
\item Les coûts de production(pour la firme en amont) correspondent à un coût marginal constant:
\begin{align*}
    c(q) &= cq \Rightarrow c^{m}(q) := \frac{\partial c}{\partial q}(q) = c, \quad c<\frac{a}{b}.
\end{align*}
\item Les coûts de distribution sont supposés nuls(pour la firme en aval). 
\item Jeux en deux étapes: à la première le producteur fixe le prix de vente $w$, et à la deuxième le distributeur fixe 
le prix à la distribution $p$ ayant observé $w$. 


\framebreak

\tikzstyle{contour} = [rectangle, rounded corners, minimum width=3cm, minimum height=1cm,text centered, draw=black,
 fill=gray!30]
 \tikzstyle{arrow} = [thick,->,>=stealth]
\begin{figure}
\begin{tikzpicture}[node distance=2cm]
    \node (start) [contour] {Producteur: $\pi_p(w, p) = wq(p) - cq(p)$.};
    \node (in1) [contour, below of=start] {Distributeur: $\pi_d(w, p) = pq(p) - wq(p)$.};
    \node (pro1) [contour, below of=in1] {Demande: $q(p)=a-bp$.};
    \draw [arrow] (start) -- (in1);
\draw [arrow] (in1) -- (pro1);
\end{tikzpicture}
\end{figure}

\framebreak

\item Résolution par induction à rebours:
\begin{enumerate}[$\star$]
\item \textbf{Dernière étape}: le distributer fixe son prix $p^*$ en maximisant son profit: 
\begin{align*}
    \pi_d(p, w) &= (a-bp)(p-w) \Rightarrow \underbrace{\frac{\partial \pi_d}{\partial p}(p^*) = 0}_{c.p.o.} \Leftrightarrow p^*(w) = \frac{a+bw}{2b}
\end{align*}
\item \textbf{Première étape}: le producteur maximise sont profit sachant que la demande est:
\begin{align*}
    q(p^*) &= a-b\left(\underbrace{\frac{a+bw}{2b}}_{=p^*(w)}\right) = \frac{a-bw}{2},
\end{align*}
et le profit à maximiser est:
\begin{align}
    \pi_p(w) &= \left(\frac{a-bw}{2}\right)(w - c) \Rightarrow \underbrace{\frac{\partial \pi_p}{\partial w}(w^*) = 0}_{c.p.o.} 
    \Leftrightarrow w^* = \frac{a}{2b} + \frac{c}{2} 
    \label{eq1}
\end{align}
d'où:
\begin{align}
p^* &:= p^*(w^*) = \frac{a+bw^*}{2b} = \frac{a}{2b} + \frac{w^*}{2} = \frac{3a}{4b} +\frac{c}{4}, \nonumber\\
q^* &= q(p^*) = a-bp^* = a-b\left(\frac{3a}{4b} +\frac{c}{4}\right)=\frac{a}{4} - \frac{bc}{4}
\label{eq2}
\end{align}
\end{enumerate}
\item Il est alors facile de vérifier qu'avec $c<\frac{a}{b}$,  que le prix $p^*$ et la quantité $q^*$ sont respectivement plus élevé, et plus bas 
que ceux à l'équilibre du monopole, qui correspond à celui où les deux activités de production et distribution sont intégrées.
\item En effet en monopole le prix d'équilibre vérifie la c.p.o. dans la maximisation du profit de monopole: 
\begin{align}
\pi_m(p) &= pq(p) - c(q(p)) = pq(p) - cq(p)= (p - c)q(p) \Rightarrow \underbrace{\frac{\partial \pi_m}{\partial p}(p^m) = 0 }_{c.p.o.} \Leftrightarrow p^m = \frac{a}{2b} + \frac{c}{2},
\label{eq3}
\end{align}
où l'on note $p^m$ le prix d'équilibre en monopole, et l'on vérifie que:
\begin{align*}
p^* - p^m &= \frac{a}{4b} - \frac{c}{4} = \frac{1}{4}\left(\frac{a}{b}- c\right) > 0 \Leftrightarrow \frac{a}{b} > c.
\end{align*}
On peut aussi vérifier que:

\begin{enumerate}[$\star$]
\item Le profit obtenu en monopole est plus élevé que le profit total sans intégration. 
\item La quantité sur le marché est plus élevée en monopole que sans intégration. 
\end{enumerate}

\item Il y a donc une forte incitation à l'intégration des deux activités au sein d'une seule firme, 
aussi bien du point de vue des firmes(profits plus élevés avec intégration), que des consommateurs(prix plus bas avec intégration).
\item Ceci est illustré numériquement dans l'exercice du TD.

\end{itemize}
\end{frame}

\begin{frame}[allowframebreaks]{Application: exercice producteur de jouet et distributeur}
\begin{enumerate}
\item On applique les résultats du modèle précédent avec ici les valeurs numériques 
sur la demande et le coût:
\begin{align*}
    q(p) &= \underbrace{360}_{=:a} - \underbrace{8}_{=:b}p, \quad c^m(q) = \underbrace{5}_{=:c}.
\end{align*}
et obtient par \eqref{eq1} et \eqref{eq2}:
\begin{align*}
    w^* &= \frac{a}{2b} + \frac{c}{2} = 25,\\
    p^* &=  \frac{3a}{4b} +\frac{c}{4} = 35,\\
    q^* &= \frac{a}{4} - \frac{bc}{4} = 80.
\end{align*}
On peut aussi calculer le profits à l'équilibre: 
\begin{align*}
\pi_d^*(w^*, p^*, q^*) &= (p^* - w^*) q^* = 800,\\
\pi_p^*(w^*, q^*) &= w^*q^* - c(q^*) = 1600.
\end{align*}
\item Dans le cas d'un intégration le prix d'équilibre est celui donné par \eqref{eq3}, et l'on calcule aussi la 
quantité et le profit d'équilibre:
\begin{align*}
p^m &=  \frac{a}{2b} + \frac{c}{2} = 25,\\
q^m &:=q(p^m) = 180 - 10p^m = 160,\\
\pi^m &:= \pi_m(p^m, q^m)= 3200.
\end{align*}
\item Cet exercice illustre ce qui a été indiqué plus haut avec: 
\begin{enumerate}[$\star$]
\item un profit total plus bas sans intégration qu'avec,
\item un prix plus élevé sans intégration qu'avec.
\item Il y a donc une incitation forte à l'intégration du fait de l'externalité 
négative produite par l'absence de coordination entre les deux firme et se traduisant 
par la double marge: le distributeur en imposant sa marge sur le prix en impose une deuxième. 
\end{enumerate}
\item Le problème de la DM peut être résolu par le biais d'ententes verticales:
\begin{enumerate}[$\star$]
    \item La question est celle de la possibilité par le producteur, dans ce cas le producteur de jouet, 
    d'éviter que le distributeur impose sa marge sur le prix, 
    cela par l'incitation ou la contrainte.
    \item Rappel des points traités en cours:
    \begin{enumerate}[$\star$]
        \item tarification en deux parties,
        \item prix de vente imposé/conseillé,
        \item quota de vente minimale.
    \end{enumerate}
\end{enumerate}
\end{enumerate}
\end{frame}

\begin{frame}[allowframebreaks]{Références}
\bibliographystyle{jpe}
\bibliography{../../../Biblio}
\end{frame}
\end{document}
