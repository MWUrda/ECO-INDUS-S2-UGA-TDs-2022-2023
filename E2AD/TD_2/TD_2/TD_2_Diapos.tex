%\documentclass[ignorenonframetext, compress, 9pt, xcolor=svgnames]{beamer} 
\input{../../../Config_diapos}
\usepackage[svgnames]{xcolor}
\usepackage{tikz}
\usetikzlibrary{shapes.geometric, arrows}
\usepackage{enumerate}   
\usepackage{multirow}
\usepackage{txfonts}
\usepackage{mathrsfs}
\usepackage{pgfplots}
\pgfplotsset{compat = newest}
\usetikzlibrary{positioning, arrows.meta}
\usepgfplotslibrary{fillbetween}
\newcommand{\A}{(0,0) ++(135:2) circle (2)}
\newcommand{\B}{(0,0) ++(45:2) circle (2)}
\DeclareMathOperator{\C}{C}
\DeclareMathOperator{\util}{u}
%\setbeamersize{text margin left=1.5em,text margin right=1.5em} 
%\setbeamersize{text margin left=1.2cm,text margin right=1.2cm} 
\setbeamersize{text margin left=1.5em,text margin right=1.5em} 
%\usepackage{xr}
%\externaldocument{Econometrie1_UGA_P2e}
  \usepackage{eso-pic}
%\newcommand\AtPagemyUpperLeft[1]{\AtPageLowerLeft{%
%\put(\LenToUnit{0.9\paperwidth},\LenToUnit{0.85\paperheight}){#1}}}
%\AddToShipoutPictureFG{
 % \AtPagemyUpperLeft{{\includegraphics[width=1.1cm,keepaspectratio]{logoUGA2020.pdf}}}
%}%

%\setbeamercolor{title}{fg=black}
%\setbeamercolor{frametitle}{fg=black}
%\setbeamercolor{section in head/foot}{fg=black}
%\setbeamercolor{author in head/foot}{bg=Brown}
%\setbeamercolor{date in head/foot}{fg=Brown}
\AtBeginSection[]
  {
    \ifnum \value{framenumber}>1
      \begin{frame}<beamer>
      \frametitle{PLAN}
      \tableofcontents[currentsection]
      \end{frame}
    \else
    \fi
  }
\setbeamertemplate{section page}
{
    \begin{centering}
    \begin{beamercolorbox}[sep=11pt,center]{part title}
    \usebeamerfont{section title}\thesection.~\insertsection\par
    \end{beamercolorbox}
    \end{centering}
}
%\titlegraphic{\includegraphics[width=1cm]{logoUGA2020.pdf}}
\title[]{ \textbf{ÉCONOMIE INDUSTRIELLE}\footnote{Responsable du cours: Alexis Garapin.}\\(\textbf{UGA, L3 E2AD, S2})}
\subtitle{TRAVAUX DIRIGÉS: TD 2 \\  DISSUASION À L'ENTRÉE II.}
\date{\today}
\author{Michal W. Urdanivia\inst{*}}
\institute{\inst{*}UGA, Facult\'e d'\'Economie, GAEL, \\
e-mail:
 \href{
     mailto:michal.wong-urdanivia@univ-grenoble-alpes.fr}{michal.wong-urdanivia@univ-grenoble-alpes.fr}}

%\titlegraphic{\includegraphics[width=1cm]{logoUGA2020.pdf}
%}

\begin{document}

%%% TIKZ STUFF
\usetikzlibrary{positioning}
\usetikzlibrary{snakes}
\usetikzlibrary{calc}
\usetikzlibrary{arrows}
\usetikzlibrary{decorations.markings}
\usetikzlibrary{shapes.misc}
\usetikzlibrary{matrix,shapes,arrows,fit,tikzmark}
\usetikzlibrary{matrix,chains,positioning,decorations.pathreplacing,arrows}
\usetikzlibrary{shapes}
\usetikzlibrary{shapes.geometric, arrows}
\tikzset{   
        every picture/.style={remember picture,baseline},
        every node/.style={anchor=base,align=center,outer sep=1.5pt},
        every path/.style={thick},
        }
\newcommand\marktopleft[1]{
    \tikz[overlay,remember picture] 
        \node (marker-#1-a) at (-.3em,.3em) {};%
}
\newcommand\markbottomright[2]{%
    \tikz[overlay,remember picture] 
        \node (marker-#1-b) at (0em,0em) {};%
}
\tikzstyle{every picture}+=[remember picture] 
\tikzstyle{mybox} =[draw=black, very thick, rectangle, inner sep=10pt, inner ysep=20pt]
\tikzstyle{fancytitle} =[draw=black,fill=red, text=white]
\tikzstyle{observed}=[draw,circle,fill=gray!50]

\begin{frame}
\titlepage
\end{frame}
\begin{frame}
 \tableofcontents
    \end{frame}
%\begin{frame}
%\frametitle{Contenu}
%\tableofcontents[pausesections, pausesubsections]
%\end{frame}

%\section{Qu'est-ce que l’économétrie ? A quoi (à qui) ça sert ?}
%\frame{\sectionpage}
%\begin{frame}
%  \tableofcontents  
%\end{frame}
\section{Exercice 1: H2O}
\frame{\sectionpage}
\begin{frame}
[allowframebreaks]{\insertsection}
\framesubtitle{Données de l'exercice}
\begin{itemize}
\item Une firme, H2O, en monopole sur un marché caractérisée par la fonction de coût: 
\begin{align}
CT(q) &= q^2 + 10q,
\label{eq1}
\end{align}
où $q\in \R^+$ représente les quantités produites en millions de $m^3$ d’eau.
\item La demande sur le marché est donnée par:
\begin{align}
p(q) &= 50-4q,
\label{eq2}
\end{align}
où $p(\cdot)$ est la demande inverse donnant le prix  en centime du bien $p=p(q)$  sur le marché pour une quantité offerte $q$.
\end{itemize}
\end{frame}

\begin{frame}
[allowframebreaks]{\insertsection}
\framesubtitle{Question 1: optimum du monopole}
\begin{itemize}
\item La firme maximise par rapport à $q$  la fonction de profit:
\begin{align}
\pi(q) &= RT(q) - CT(q),
\label{eq3}
\end{align}
où $RT(q) \coloneqq p(q)q$ est la recette de la firme. 
\item Le choix optimal $q^*$  est donc défini par:
\begin{align}
q^* &=\argmax_q \pi(q) \Rightarrow \underbrace{\frac{\partial RT(q)}{\partial q}(q^*) -  \frac{\partial CT(q)}{\partial q}(q^*) = 0}_{\text{c.p.o.}}, 
\label{eq3}
\end{align}
où encore:
\begin{align*}
\underbrace{R_m(q^*)}_{\substack{\text{Recette} \\ \text{marginale}}} &= \underbrace{C_m(q^*)}_{\substack{\text{Coût} \\ \text{marginal}}}, 
\end{align*}
avec les définitions $R_m(q)\coloneqq \frac{\partial RT(q)}{\partial q}(q)$,  $C_m(q)\coloneqq \frac{\partial CT(q)}{\partial q}(q)$.

\item Nous avons avec \eqref{eq1} et \eqref{eq2},
\begin{align*}
RT(q) =  50q-4q^2 \Rightarrow R_m(q) = 50-8q, \quad \text{et} \quad C_m(q) = 2q + 10,
\end{align*}
de sorte que la condition dans \eqref{eq3} donne:
\begin{align*}
\underbrace{50-8q^*}_{R_m(q^*)} = \underbrace{2q^* + 10}_{C_m(q^*)}  &\Rightarrow q^* = 4,
\end{align*}
d'où $p^* = p(q^*) =  34$, $\pi^* = \pi(q^*) = 80$.
\end{itemize}
\end{frame}

\begin{frame}
[allowframebreaks]{\insertsection}
\framesubtitle{Question 3: stratégie de prix limite}
\begin{itemize}
\item Un entrant potentiel désire pénétrer le marché en vendant 5 unités au coût unitaire constant de 20 euros
\item \textbf{\underline{Question}}: à quel niveau la firme installée peut-elle fixer son prix
pour rendre l’entrée non profitable ? Ce prix est le \textbf{\underline{prix limite.}}
\item \textbf{\underline{Réponse}}: ce prix, $p_L$ prend la forme,
\begin{align}
p_L &< c_e + \abs{a}q_e,
\label{eq4}
\end{align}
où:
\begin{enumerate}[$\cdot$]
\item $a<0$ est le paramètre d'une fonction de demande inverse linéaire du type $p(Q) = aQ + b$, pour $b>0$, et $Q$ étant la quantité totale offerte sur le marché
(ici: $a\equiv -4$, $b \equiv 50$),
\item $c_e$ et $q_e$ sont respectivement le coût unitaire de l'entrant potentiel, et la  quantité potentiellement offerte.
\end{enumerate}
\item Par conséquent $p_L$ vérifie,
\begin{align}
p_L &< \underbrace{20}_{q_e}+  \underbrace{4}_{\abs{a}} \times \underbrace{20}_{c_e} = 40,
\label{eq5}
\end{align}
et on note que le prix de monopole vérifie \eqref{eq5} et est un prix limite possible pour la firme en place.
\item  \textbf{\underline{Explication}}(Voir cours: Partie 1. Stratégies anticoncurrentielles, section 3, en particulier slides 35-38):
\begin{enumerate}[$\cdot$]
\item Pour que l’entrée ne soit pas profitable, il faut que le prix du
marché, à l’issue de l’entrée, soit inférieur au coût de production de
l’entrant potentiel.
\item  Prix du marché à l’issue de l’entrée:
\begin{align}
p(Q) &= aQ + b = a(q_F + q_e ) + b,
\label{eq6}
\end{align}
où $q_F$ est la quantité offerte par la firme en place.
\item Le prix du marché doit être inférieur au coût de l’entrant, donc par \eqref{eq6}:
\begin{align}
p < c_e &\Leftrightarrow  a(q_F + q_e ) + b < c_e,
\label{eq7}
\end{align}
\item D'autre part si la firme en place offre $q_F$ on a un prix $p_F$ et une quantité $q_F$ donnés par:
\begin{align}
p_F = p_F(q_F) = aq_F+ b  &\Rightarrow q_F = \frac{1}{a}(p_F - b),
\label{eq8}
\end{align}
\item En utilisant \eqref{eq7} et \eqref{eq8} on obtient:
\begin{align*}
a\left( \frac{1}{a}(p_F - b) +  q_e \right) + b < c_e &\Leftrightarrow  p_F < c_e - aq_e \Leftrightarrow p_F < c_e + \abs{a}q_e.
\end{align*}
\end{enumerate}
\end{itemize}
\end{frame}

\begin{frame}
[allowframebreaks]{\insertsection}
\framesubtitle{Question 4: rentabilité de la firme en place en cas d'entrée}
\begin{itemize}
\item La production totale est alors égale à celle de l’entrant ($q_e = 5$) plus celle de la firme en place qu'elle fixe  en utilisant \eqref{eq8} avec $p_F = 34$ qui est le prix de monopole:
\begin{align*}
q_F &=  -\frac{1}{\underbrace{4}_{a}}(\underbrace{34}_{p_F} - \underbrace{50}_{b}) = 4,
\end{align*}
d'où $Q = q_e+q_F = 9$ et un prix s'établissant à $p(9) = 50 - 4\times 9 = 14$.
\end{itemize}
\end{frame}

%\begin{frame}[allowframebreaks]{Références}
%\bibliographystyle{jpe}
%\bibliography{../../../Biblio}
%\end{frame}

\end{document}
