\documentclass[notes, ignorenonframetext, compress, 9pt, xcolor=svgnames, aspectratio=169]{beamer} 
\input{../../../Config_diapos_3}
\usepackage[svgnames]{xcolor}
\usepackage{tikz}
\usetikzlibrary{shapes.geometric, arrows}
\usepackage{enumerate}   
\usepackage{multirow}
\usepackage{txfonts}
\usepackage{mathrsfs}
\usepackage{pgfplots}
\pgfplotsset{compat = newest}
\usetikzlibrary{positioning, arrows.meta}
\usepgfplotslibrary{fillbetween}
\newcommand{\A}{(0,0) ++(135:2) circle (2)}
\newcommand{\B}{(0,0) ++(45:2) circle (2)}
\DeclareMathOperator{\C}{C}
\DeclareMathOperator{\util}{u}
%\setbeamersize{text margin left=1.5em,text margin right=1.5em} 
%\setbeamersize{text margin left=1.2cm,text margin right=1.2cm} 
\setbeamersize{text margin left=1.5em,text margin right=1.5em} 
%\usepackage{xr}
%\externaldocument{Econometrie1_UGA_P2e}
  \usepackage{eso-pic}
%\newcommand\AtPagemyUpperLeft[1]{\AtPageLowerLeft{%
%\put(\LenToUnit{0.9\paperwidth},\LenToUnit{0.85\paperheight}){#1}}}
%\AddToShipoutPictureFG{
 % \AtPagemyUpperLeft{{\includegraphics[width=1.1cm,keepaspectratio]{logoUGA2020.pdf}}}
%}%

%\setbeamercolor{title}{fg=black}
%\setbeamercolor{frametitle}{fg=black}
%\setbeamercolor{section in head/foot}{fg=black}
%\setbeamercolor{author in head/foot}{bg=Brown}
%\setbeamercolor{date in head/foot}{fg=Brown}
\AtBeginSection[]
  {
    \ifnum \value{framenumber}>1
      \begin{frame}<beamer>
      \frametitle{Outline}
      \tableofcontents[currentsection]
      \end{frame}
    \else
    \fi
  }
\setbeamertemplate{section page}
{
    \begin{centering}
    \begin{beamercolorbox}[sep=11pt,center]{part title}
    \usebeamerfont{section title}\thesection.~\insertsection\par
    \end{beamercolorbox}
    \end{centering}
}

%\titlegraphic{\includegraphics[width=1cm]{logoUGA2020.pdf}}
%\titlegraphic{\includegraphics[width=1cm]{logoUGA2020.pdf}}
\title[]{ \textbf{ÉCONOMIE INDUSTRIELLE}\footnote{Responsable du cours: Alexis Garapin.}\\(\textbf{UGA, L3 E2AD, S2})}
\subtitle{TRAVAUX DIRIGÉS: TD 3 \\   JEUX RÉPÉTÉS ET COLLUSION}
\date{\today}
\author{Michal W. Urdanivia\inst{*}}
\institute{\inst{*}UGA, Facult\'e d'\'Economie, GAEL, \\
e-mail:
 \href{
     mailto:michal.wong-urdanivia@univ-grenoble-alpes.fr}{michal.wong-urdanivia@univ-grenoble-alpes.fr}}

%\titlegraphic{\includegraphics[width=1cm]{logoUGA2020.pdf}
%}

\begin{document}

%%% TIKZ STUFF
\usetikzlibrary{positioning}
\usetikzlibrary{snakes}
\usetikzlibrary{calc}
\usetikzlibrary{arrows}
\usetikzlibrary{decorations.markings}
\usetikzlibrary{shapes.misc}
\usetikzlibrary{matrix,shapes,arrows,fit,tikzmark}
\usetikzlibrary{matrix,chains,positioning,decorations.pathreplacing,arrows}
\usetikzlibrary{shapes}
\usetikzlibrary{shapes.geometric, arrows}
\tikzset{   
        every picture/.style={remember picture,baseline},
        every node/.style={anchor=base,align=center,outer sep=1.5pt},
        every path/.style={thick},
        }
\newcommand\marktopleft[1]{
    \tikz[overlay,remember picture] 
        \node (marker-#1-a) at (-.3em,.3em) {};%
}
\newcommand\markbottomright[2]{%
    \tikz[overlay,remember picture] 
        \node (marker-#1-b) at (0em,0em) {};%
}
\tikzstyle{every picture}+=[remember picture] 
\tikzstyle{mybox} =[draw=black, very thick, rectangle, inner sep=10pt, inner ysep=20pt]
\tikzstyle{fancytitle} =[draw=black,fill=red, text=white]
\tikzstyle{observed}=[draw,circle,fill=gray!50]



\begin{frame}
\titlepage
\end{frame}
\begin{frame}
 \tableofcontents
    \end{frame}
%\begin{frame}
%\frametitle{Contenu}
%\tableofcontents[pausesections, pausesubsections]
%\end{frame}

%\section{Qu'est-ce que l’économétrie ? A quoi (à qui) ça sert ?}
%\frame{\sectionpage}
%\begin{frame}
%  \tableofcontents  
%\end{frame}
\section{Rappels de cours}
\frame{\sectionpage}
\begin{frame}
[allowframebreaks]{\insertsection}
\framesubtitle{Le modèle de Cournot de base \\}
\begin{itemize}
    \item Il s'agit d'un modèle de concurrence à la Cournot. 
        \item Chaque firme a une fonction objectif qui est son profit: 
        \begin{align*}
            \pi_i(q_i, q_j) &= P(Q)q_i -c_i(q_i),  \ i, j= 1, 2, \ i\neq j.
        \end{align*}
        où $P(Q)$ est la fonction de demande inverse avec $Q = q_1 + q_2$ est $c_i(q_i)$ 
        est la fonction de coût de la firme $i$.
        \item La variable de décision est la quantité à produire $q_i$. 
        \item L'équilibre est une paire $(q_1^*, q_2^*)$ qui est un équilibre de Nash dans un jeu d'information complète.
        \item En outre le modèle de base suppose les formes suivantes pour $c_i(\cdot)$ et $P(\cdot)$:
        \begin{align*}
        c_i(q_i) &\coloneqq cq_i \Rightarrow  \underbrace{c^m_i(q_i)}_{\substack{\text{Coût}\\ \text{Marginal}}} \coloneqq\frac{\partial c}{\partial q_i}(q_i) = c, \ c\in\R_{+}^*, \\
        P(Q) &\coloneqq a - bQ, \ a, b  \in\R_{+}^*\times \in\R_{+}^*.
        \end{align*}
        
        \item  Le choix optimal de la firme $i$ est donné par sa \textbf{fonction de meilleure réponse}, $q_i^{mr}(q_j)$  qui est définie implicitement comme solution du problème:
        \begin{align*}
        q_i^* &= \argmax_{q_i}\pi_i(q_i, q_j) \eqqcolon q_i^{mr}(q_j) 
        \end{align*}
        où $\pi(q_i, q_j)$ est la fonction de profit de la firme $i$:
        \begin{align*}
        \pi_i(q_i, q_j) &\coloneqq P(Q) q_i - c_i(q_i).
        \end{align*}
        \item  L'équilibre  $(q_1^*, q_2^*)$ est alors obtenu comme solution du système:
       \begin{align*}
       q_i^* =  q_i^{mr}(q_j^*),  \ i, j = 1, 2; \ i\neq j,
       \end{align*}
       Autrement dit, chaque firme joue sa meilleure réponse.
        
        \item Avec les fonctions $c_i(q_i) \coloneqq cq_i$ et $P(Q) \coloneqq a - bQ$, $q_i^{mr}(q_j)$ est explicitée à partir de la condition du 1er ordre associée à la maximisation de $\pi_i(\cdot)$ par rapport à $q_i$:
    \begin{align*}
        \frac{\partial \pi}{\partial q_i}\left(q_i^*, q_j\right)=0   &\Leftrightarrow   a - 2bq_i^* - bq_j = c  \Leftrightarrow q_i^* = \frac{(a-c)}{2b} - \frac{q_j}{2} \eqqcolon q_i^{mr}(q_j). 
    \end{align*}
   \item On calcule l'équilibre comme solution en $(q_1^*, q_2^*)$  du système:
   \begin{align*}
   \left\{
            \begin{array}{l}
            q_1^* = q_1^{mr}(q_2^*) \\
            q_2^* = q_2^{mr}(q_1^*) 
            \end{array}\right.
        \end{align*}
 ce qui donne avec les fonction de meilleure réponse $q_1^{mr}(q_2) \coloneqq \frac{(a-c)}{2b} - \frac{q_2}{2}$, et
  $q_2^{mr}(q_1) \coloneqq \frac{(a-c)}{2b} - \frac{q_1}{2}$.
 \begin{align}
    q_1^* &= q_2^*  = \frac{a-c}{3b}.
    \label{eq1}
  \end{align}
  \item Et  on calcule aussi les profits et prix du bien sur le marché:
  \begin{align*}
      p^{*} = \frac{a}{3} +\frac{2c}{3}&, \  \pi_1^{*} =  \pi_2^{*} = \frac{(a-c)^2}{9b},
      \end{align*}
\end{itemize}
\end{frame}

\begin{frame}
[allowframebreaks]{\insertsection}
\framesubtitle{Entente\\}

    \begin{itemize}
        \item Les deux firmes fixent la quantité de monopole notée $q^{ca}$ qu'elles décident en maximisant un profit de Monopole.
        \item Pour un niveau décidé de produit $q^{ca}$ chacune produit alors $q_i^{ca} = \frac{q^{ca}}{2}$, $i=1, 2$.
        \item Pour obtenir  $q^{ca}$  on considère le profit donné par,
        \begin{align*}
            \pi(q^{ca}) &= P(q^{ca})q^{ca} - cq^{ca} = (a-bq^{ca})q^{ca} - cq^{ca}
        \end{align*}
        \item La quantité optimale/de monopole $q^{ca^*}$ est obtenue comme:
        \begin{align*}
            q^{ca^*} &=\argmin_{q^{ca}}  \pi(q) \Rightarrow \frac{\partial \pi}{\partial q}( q^{ca^*}) = 0 
            \Leftrightarrow  q^{ca^*} = \frac{(a-c)}{2b}.
        \end{align*}
        D'où le prix d'équilibre et le profit du cartel:
        \begin{align*}
        p^{ca^*} &\coloneqq P(q^{ca^*}) = a - b\left(q^{ca^*}\right) = \frac{(a-c)}{2},\\
        \pi ^{ca^*} &\coloneqq p^{ca^*}  q^{ca^*}  - c q^{ca^*}  =  \frac{(a-c)^2}{4b}.
        \end{align*}
        Le profit de la firme $i$ étant alors:
        \begin{align*}
        \pi_i ^{ca^*} &\coloneqq \frac{\pi ^{ca^*}}{2} = \frac{(a-c)^2}{8b}.
        \end{align*}
    \end{itemize}
    \end{frame}

\begin{frame}
[allowframebreaks]{\insertsection}
\framesubtitle{Déviation sur une période\\}
\begin{itemize}
\item Supposons que la firme $i$ dévie/triche par rapport à l'entente.
\item Dans ce cas alors que $j$ joue la quantité de l'entente  $q_j^{ca^*} \coloneqq \frac{q^{ca^*}}{2} =  \frac{(a-c)}{4b}$, $i$ joue sa meileure réponse 
et fait le choix $q_i^{d^*}$ par:
\begin{align*}
q_i^{d^*} &\coloneqq = q_i^{mr}(q_j^{ca^*} ) = \frac{(a-c)}{2b} - \frac{q_j^{ca^*} }{2} = \frac{(a-c)}{2b} - \frac{(a-c)}{8b} = \frac{3(a-c)}{8b},
\end{align*}
ce qui lui donne un profit,
\begin{align*}
\pi_i^{d^*} &= \frac{9(a-c)^2}{64b}.
\end{align*}
\end{itemize}
    \end{frame}
    
\begin{frame}
[allowframebreaks]{\insertsection}
\framesubtitle{Stabilité de l’entente\\}
\begin{itemize}
\item En fait le jeu consistant à tricher ou s'entendre s'apparente au
 \textbf{\underline{jeu du prisonnier}} avec la représentation de forme normale et extensives suivantes:



		\begin{figure}[!h]
\begin{tabular}{*4c} 
	& &\multicolumn{2}{c}{Firme 2}\\
	& & Tricher &Cartel \\ \cline{3-4}
	\multirow{2}{*}{Firme 1}& Tricher & \multicolumn{1}{|c|}{$\underbrace{\frac{(a-c)^2}{9b}, \frac{(a-c)^2}{9b}}_{\text{Équilibre de Nash}}$} & \multicolumn{1}{c|}{$\frac{9(a-c)^2}{64b},\frac{3(a-c)^2}{32b}$} \\[10pt] \cline{3-4} 
	& Cartel & \multicolumn{1}{|c|}{$\frac{3(a-c)^2}{32b}, \frac{9(a-c)^2}{64b}$} &\multicolumn{1}{c|}{$\frac{(a-c)^2}{8b}, \frac{(a-c)^2}{8b}$}\\[10pt] \cline{3-4}
	\end{tabular}
    \caption{Représentation de la concurrence à la Cournot sous forme normale.}
\end{figure}

\begin{figure}
    \begin{center}
   \small
   \begin{tikzpicture}[thin,
     level 1/.style={sibling distance=60mm},
     level 2/.style={sibling distance=30mm},
     level 3/.style={sibling distance=15mm},
     every circle node/.style={minimum size=1.5mm,inner sep=0mm}]
     
     \node[circle,draw,label=above: Firme 1] (root) {}
       child { node [circle,fill,label=above: Firme 2] {}
         child { 
           node {$\left(\frac{(a-c)^2}{8b}, \frac{(a-c)^2}{8b}\right)$}
           edge from parent
             node[left] {Cartel}}
         child { 
           node {$\left(\frac{3(a-c)^2}{32b}, \frac{9(a-c)^2}{64b}\right)$}
           edge from parent
             node[right] {Triche}}
         edge from parent
           node[left] {Cartel}}
       child { node [circle,fill,label=above: Firme 2] {}
       child { 
        node {$\left(\frac{9(a-c)^2}{64b},\frac{3(a-c)^2}{32b}\right)$}
        edge from parent
          node[left] {Cartel}}
          child { 
            node {$\left(\frac{(a-c)^2}{9b}, \frac{(a-c)^2}{9b}\right)$}
            edge from parent
              node[right] {Triche}}
          edge from parent
            node[right] {Triche}};
   \end{tikzpicture}
   \end{center}
   \caption{Représentation de la concurrence à la Cournot sous forme extensive}
 \end{figure}
    \end{itemize}
\end{frame}

\begin{frame}
[allowframebreaks]{\insertsection}
\framesubtitle{Jeu/concurrence à la Cournot infiniment répété et Actualisation\\}
   \begin{itemize}
        \item Supposons que le jeu se répète deux fois: la firme $i$ choisit 
        la quantité $q_{it}$ à la période $t$, pour $i=1, 2$, et $t=1, 2$.
        \item \textbf{\underline{Quel est l'équilibre en sous-jeu parfait?}} 
        \item On résout le jeu(par induction) à rebours:
        \begin{enumerate}[-]
            \item En $t=2$ l'unique équilibre de Nash est $(triche, triche)$.
            \item En $t=1$ l'unique équilibre de Nash est $(triche, triche)$.
            \item Par conséquent, l'unique équilibre en sous-jeux parfait est $\left( 
                (triche, triche), (triche, triche)
            \right)$.
        \end{enumerate}
        \item \textbf{\underline{Autres questions}} : 
        \begin{enumerate}[-]
         \item qu'en est-il de $\left((cartel,cartel), (cartel,cartel)\right)$? 
         \item qu'en est-il de $\left((cartel, triche), (cartel, triche)\right)$?
         \item qu'en est-il de: la firm 1 joue $(cartel; triche \ \text{si} \ triche, cartel  \ \text{si} \ cartel)$,
          et la firme 2 joue $(cartel; triche \ \text{si} \ triche, cartel  \ \text{si} \ cartel)$?
        \item du jeu à 3,..., $N$ périodes?
\end{enumerate}
\end{itemize}
\end{frame}
\begin{frame}
[allowframebreaks]{\insertsection}
\framesubtitle{Cournot infiniment\\}

\begin{itemize}
    \item On note comme précédemment $q^{ca}_i$ la quantité de cartel pour la firme $i$(maximisant le profit des deux firmes), et $q_i^*$ 
    la quantité en concurrence à la Cournot.
    \item Nous avons le résultat suivant:  
    \begin{proposition}
        Si le taux d'actualisation $\delta$ est "suffisamment"  élevé 
         alors les stratégies suivantes constituent des équilibre de sous-jeu parfaits pour  
         le jeu de Cournot infiniment répété:  
         \begin{enumerate}[(a)]
             \item En $t$, la firme $i$ joue  $q_{it} =q^{ca}_i$ si $q_{j,t-1} =q^{ca}_j$ pour $j=1$ et $j=2$. 
             \item Jouer $q^*_i$ si $q_{j,t-1} \neq q^{ca}_j$ pour soit $j=1$ ou $j=2$.
         \end{enumerate}
    \end{proposition}
    \item La firme $i$ coopère tant que $j$ coopère.
    \item Une fois que $j$ triche $i$ produit la quantité d'équilibre de Nash-Cournot 
    pour toutes les périodes suivantes: \textbf{Nash reversion}.
    \item “Grim strategy”: pas de deuxième chance.
    \item Pour démontrer que ces strategies constituent des équilibres en sous-jeu parfaits il   
    faut obtenir des conditions qui "prescrivent" que la meilleure réponse 
    de la firme $i$ étant donné celle de la firme $j$ est aussi la meilleure réponse 
    dans chaque sous-jeu.
\end{itemize}

\end{frame}

\begin{frame}[allowframebreaks]{\insertsection}
\framesubtitle{Éléments de démonstration}
\begin{itemize}
    \item Pour la firme $i=1, 2$, deux types de sous-jeu sont à considérer:
    \item \textbf{\underline{Sous-jeu de type 1}}:
    Après une période où un des joueurs à triché (dont soit $i$, ou $j=1, 2$, $i\neq j$):
    \begin{enumerate}[-]
        \item La stratégie proposée indique de jouer $q_i^*$ pour toujours étant donné que $j$ joue aussi 
        cette stratégie.  
        \item C'est un équilibre de Nash du sous-jeu: jouer "$q_i^*$ pour toujours" est la meilleure 
        réponse à la stratégie de $j$ de jouer "$q_j^*$ pour toujours".
        \item Ceci vérifie les critères d'un équilibre de sous-jeu parfait.
    \end{enumerate}
    \framebreak
    \item \textbf{\underline{Sous-jeu de type 2}}: Après une période sans triche.
    \begin{enumerate}[-]
        \item La stratégie proposée indique de coopérer et jouer $q_i^{ca}$, avec le profit actualisé de $\frac{\pi_i^{ca}}{1-\delta}$
        \item La meilleure stratégie alternative est de jouer $q_i^{mr}(q_j^{ca}) =: q_i^d$ à la période en cours,
         mais ceci entraîne $q_j =  q_j^*$ pour toujours. Le profit actualisé est ici 
          $\pi^{d^*_i}  + \delta\left(\frac{\pi_i^*}{(1-\delta)}\right)$.
         \item Pour que $q_i^{ca}$ soit un équilibre de Nash de ce sous-jeu, il est nécessaire que,
         \begin{align*}
            \underbrace{\frac{\pi_i^{ca^*}}{1-\delta}}_{\text{profits sous coopération}} &> \underbrace{\pi^{d^*_i} + \delta\left(\frac{\pi_i^*}{(1-\delta)}\right)}_{\text{profits sous déviation/triche}}\\
            &\Leftrightarrow \delta > \frac{\pi^{d^*_i} - \pi_i^{ca^*} }{\pi^{d^*_i}  - \pi_i^*},
            \end{align*}
         et avec les expressions de $\pi_i^{ca^*}$, $\pi^{d^*_i} $, et $\pi_i^*$   on obtient que : $\delta > \frac{9}{17}\approx 0.529$.
    \end{enumerate}
    \item Par conséquent, la \textbf{Nash reversion} indique une meilleure réponse dans ces deux sous-jeux 
    si est "suffisamment élevé" avec $\delta >\frac{9}{17}$. 
    \item Dans ce cas la \textbf{Nash reversion} constitue un équilibre de sous-jeu parfait.
\end{itemize}

\end{frame}
\end{document}
