%\documentclass[ignorenonframetext, compress, 9pt, xcolor=svgnames]{beamer} 
\input{../../../Config_diapos}
\usepackage[svgnames]{xcolor}
\usepackage{tikz}
\usetikzlibrary{shapes.geometric, arrows}
\usepackage{enumerate}   
\usepackage{multirow}
\usepackage{txfonts}
\usepackage{mathrsfs}
\usepackage{pgfplots}
\pgfplotsset{compat = newest}
\usetikzlibrary{positioning, arrows.meta}
\usepgfplotslibrary{fillbetween}
\newcommand{\A}{(0,0) ++(135:2) circle (2)}
\newcommand{\B}{(0,0) ++(45:2) circle (2)}
\DeclareMathOperator{\C}{C}
\DeclareMathOperator{\util}{u}
%\setbeamersize{text margin left=1.5em,text margin right=1.5em} 
%\setbeamersize{text margin left=1.2cm,text margin right=1.2cm} 
\setbeamersize{text margin left=1.5em,text margin right=1.5em} 
%\usepackage{xr}
%\externaldocument{Econometrie1_UGA_P2e}
  \usepackage{eso-pic}
%\newcommand\AtPagemyUpperLeft[1]{\AtPageLowerLeft{%
%\put(\LenToUnit{0.9\paperwidth},\LenToUnit{0.85\paperheight}){#1}}}
%\AddToShipoutPictureFG{
 % \AtPagemyUpperLeft{{\includegraphics[width=1.1cm,keepaspectratio]{logoUGA2020.pdf}}}
%}%

%\setbeamercolor{title}{fg=black}
%\setbeamercolor{frametitle}{fg=black}
%\setbeamercolor{section in head/foot}{fg=black}
%\setbeamercolor{author in head/foot}{bg=Brown}
%\setbeamercolor{date in head/foot}{fg=Brown}
\AtBeginSection[]
  {
    \ifnum \value{framenumber}>1
      \begin{frame}<beamer>
      \frametitle{PLAN}
      \tableofcontents[currentsection]
      \end{frame}
    \else
    \fi
  }
\setbeamertemplate{section page}
{
    \begin{centering}
    \begin{beamercolorbox}[sep=11pt,center]{part title}
    \usebeamerfont{section title}\thesection.~\insertsection\par
    \end{beamercolorbox}
    \end{centering}
}
%\titlegraphic{\includegraphics[width=1cm]{logoUGA2020.pdf}}
\title[]{ \textbf{ÉCONOMIE INDUSTRIELLE}\footnote{Responsable du cours: Alexis Garapin.}\\(\textbf{UGA, L3 E2AD, S2})}
\subtitle{TRAVAUX DIRIGÉS: TD 3 \\  JEUX RÉPÉTÉS ET COLLUSION.}
\date{\today}
\author{Michal W. Urdanivia\inst{*}}
\institute{\inst{*}UGA, Facult\'e d'\'Economie, GAEL, \\
e-mail:
 \href{
     mailto:michal.wong-urdanivia@univ-grenoble-alpes.fr}{michal.wong-urdanivia@univ-grenoble-alpes.fr}}

%\titlegraphic{\includegraphics[width=1cm]{logoUGA2020.pdf}
%}

\begin{document}

%%% TIKZ STUFF
\usetikzlibrary{positioning}
\usetikzlibrary{snakes}
\usetikzlibrary{calc}
\usetikzlibrary{arrows}
\usetikzlibrary{decorations.markings}
\usetikzlibrary{shapes.misc}
\usetikzlibrary{matrix,shapes,arrows,fit,tikzmark}
\usetikzlibrary{matrix,chains,positioning,decorations.pathreplacing,arrows}
\usetikzlibrary{shapes}
\usetikzlibrary{shapes.geometric, arrows}
\tikzset{   
        every picture/.style={remember picture,baseline},
        every node/.style={anchor=base,align=center,outer sep=1.5pt},
        every path/.style={thick},
        }
\newcommand\marktopleft[1]{
    \tikz[overlay,remember picture] 
        \node (marker-#1-a) at (-.3em,.3em) {};%
}
\newcommand\markbottomright[2]{%
    \tikz[overlay,remember picture] 
        \node (marker-#1-b) at (0em,0em) {};%
}
\tikzstyle{every picture}+=[remember picture] 
\tikzstyle{mybox} =[draw=black, very thick, rectangle, inner sep=10pt, inner ysep=20pt]
\tikzstyle{fancytitle} =[draw=black,fill=red, text=white]
\tikzstyle{observed}=[draw,circle,fill=gray!50]

\begin{frame}
\titlepage
\end{frame}
\begin{frame}
 \tableofcontents
    \end{frame}
%\begin{frame}
%\frametitle{Contenu}
%\tableofcontents[pausesections, pausesubsections]
%\end{frame}

%\section{Qu'est-ce que l’économétrie ? A quoi (à qui) ça sert ?}
%\frame{\sectionpage}
%\begin{frame}
%  \tableofcontents  
%\end{frame}
\section{Rappels de cours}
\frame{\sectionpage}
\begin{frame}
[allowframebreaks]{\insertsection}
\framesubtitle{Le modèle de Cournot de base \\}
\begin{itemize}
    \item Il s'agit d'un modèle de concurrence à la Cournot. 
        \item Chaque firme a une fonction objectif qui est son profit: 
        \begin{align*}
            \pi_i(q_i, q_j) &= P(Q)q_i -c_i(q_i),  \ i, j= 1, 2, \ i\neq j.
        \end{align*}
        où $P(Q)$ est la fonction de demande inverse avec $Q = q_1 + q_2$ est $c_i(q_i)$ 
        est la fonction de coût de la firme $i$.
        \item La variable de décision est la quantité à produire $q_i$. 
        \item L'équilibre est une paire $(q_1^*, q_2^*)$ qui est un équilibre de Nash dans un jeu d'information complète.
        \item En outre le modèle de base suppose les formes suivantes pour $c_i(\cdot)$ et $P(\cdot)$:
        \begin{align*}
        c_i(q_i) &\coloneqq cq_i \Rightarrow  \underbrace{c^m_i(q_i)}_{\substack{\text{Coût}\\ \text{Marginal}}} \coloneqq\frac{\partial c}{\partial q_i}(q_i) = c, \ c\in\R_{+}^*, \\
        P(Q) &\coloneqq a - bQ, \ a, b  \in\R_{+}^*\times \in\R_{+}^*.
        \end{align*}
        
        \item  Le choix optimal de la firme $i$ est donné par sa \textbf{fonction de meilleure réponse}, $q_i^{mr}(q_j)$  qui est définie implicitement comme solution du problème:
        \begin{align*}
        q_i^* &= \argmax_{q_i}\pi_i(q_i, q_j) \eqqcolon q_i^{mr}(q_j) 
        \end{align*}
        où $\pi(q_i, q_j)$ est la fonction de profit de la firme $i$:
        \begin{align*}
        \pi_i(q_i, q_j) &\coloneqq P(Q) q_i - c_i(q_i).
        \end{align*}
        \item  L'équilibre  $(q_1^*, q_2^*)$ est alors obtenu comme solution du système:
       \begin{align*}
       q_i^* =  q_i^{mr}(q_j^*),  \ i, j = 1, 2; \ i\neq j,
       \end{align*}
       Autrement dit, chaque firme joue sa meilleure réponse.
        
        \item Avec les fonctions $c_i(q_i) \coloneqq cq_i$ et $P(Q) \coloneqq a - bQ$, $q_i^{mr}(q_j)$ est explicitée à partir de la condition du 1er ordre associée à la maximisation de $\pi_i(\cdot)$ par rapport à $q_i$:
    \begin{align*}
        \frac{\partial \pi}{\partial q_i}\left(q_i^*, q_j\right)=0   &\Leftrightarrow   a - 2bq_i^* - bq_j = c  \Leftrightarrow q_i^* = \frac{(a-c)}{2b} - \frac{q_j}{2} \eqqcolon q_i^{mr}(q_j). 
    \end{align*}
   \item On calcule l'équilibre comme solution en $(q_1^*, q_2^*)$  du système:
   \begin{align*}
   \left\{
            \begin{array}{l}
            q_1^* = q_1^{mr}(q_2^*) \\
            q_2^* = q_2^{mr}(q_1^*) 
            \end{array}\right.
        \end{align*}
 ce qui donne avec les fonction de meilleure réponse $q_1^{mr}(q_2) \coloneqq \frac{(a-c)}{2b} - \frac{q_2}{2}$, et
  $q_2^{mr}(q_1) \coloneqq \frac{(a-c)}{2b} - \frac{q_1}{2}$.
 \begin{align}
    q_1^* &= q_2^*  = \frac{a-c}{3b}.
    \label{eq1}
  \end{align}
  \item Et  on calcule aussi les profits et prix du bien sur le marché:
  \begin{align*}
      p^{*} = \frac{a}{3} +\frac{2c}{3}&, \  \pi_1^{*} =  \pi_2^{*} = \frac{(a-c)^2}{9b},
      \end{align*}
\end{itemize}
\end{frame}

\begin{frame}
[allowframebreaks]{\insertsection}
\framesubtitle{Entente\\}

    \begin{itemize}
        \item Les deux firmes fixent la quantité de monopole notée $q^{ca}$ qu'elles décident en maximisant un profit de Monopole.
        \item Pour un niveau décidé de produit $q^{ca}$ chacune produit alors $q_i^{ca} = \frac{q^{ca}}{2}$, $i=1, 2$.
        \item Pour obtenir  $q^{ca}$  on considère le profit donné par,
        \begin{align*}
            \pi(q^{ca}) &= P(q^{ca})q^{ca} - cq^{ca} = (a-bq^{ca})q^{ca} - cq^{ca}
        \end{align*}
        \item La quantité optimale/de monopole $q^{ca^*}$ est obtenue comme:
        \begin{align*}
            q^{ca^*} &=\argmin_{q^{ca}}  \pi(q) \Rightarrow \frac{\partial \pi}{\partial q}( q^{ca^*}) = 0 
            \Leftrightarrow  q^{ca^*} = \frac{(a-c)}{2b}.
        \end{align*}
        D'où le prix d'équilibre et le profit du cartel:
        \begin{align*}
        p^{ca^*} &\coloneqq P(q^{ca^*}) = a - b\left(q^{ca^*}\right) = \frac{(a-c)}{2},\\
        \pi ^{ca^*} &\coloneqq p^{ca^*}  q^{ca^*}  - c q^{ca^*}  =  \frac{(a-c)^2}{4b}.
        \end{align*}
        Le profit de la firme $i$ étant alors:
        \begin{align*}
        \pi_i ^{ca^*} &\coloneqq \frac{\pi ^{ca^*}}{2} = \frac{(a-c)^2}{8b}.
        \end{align*}
    \end{itemize}
    \end{frame}

\begin{frame}
[allowframebreaks]{\insertsection}
\framesubtitle{Déviation sur une péridoe\\}
\begin{itemize}
\item Supposons que la firme $i$ dévie/triche par rapport à l'entente.
\item Dans ce cas alors que $j$ joue la quantité de l'entente  $q_j^{ca^*} \coloneqq \frac{q^{ca^*}}{2} =  \frac{(a-c)}{4b}$, $i$ joue sa meileure réponse 
et fait le choix $q_i^{d^*}$ par:
\begin{align*}
q_i^{d^*} &\coloneqq = q_i^{mr}(q_j^{ca^*} ) = \frac{(a-c)}{2b} - \frac{q_j^{ca^*} }{2} = \frac{(a-c)}{2b} - \frac{(a-c)}{8b} = \frac{3(a-c)}{8b},
\end{align*}
ce qui lui donne un profit,
\begin{align*}
\pi_i^{d^*} &= \frac{9(a-c)^2}{64b}.
\end{align*}
\end{itemize}
    \end{frame}
\end{document}
