\documentclass[notes, ignorenonframetext, compress, 9pt, xcolor=svgnames, aspectratio=169]{beamer} 
\input{../../../Config_diapos_2}
\usepackage[svgnames]{xcolor}
\usepackage{tikz}
\usetikzlibrary{shapes.geometric, arrows}
\usepackage{enumerate}   
\usepackage{multirow}
\usepackage{txfonts}
\usepackage{mathrsfs}
\usepackage{pgfplots}
\pgfplotsset{compat = newest}
\usetikzlibrary{positioning, arrows.meta}
\usepgfplotslibrary{fillbetween}
\newcommand{\A}{(0,0) ++(135:2) circle (2)}
\newcommand{\B}{(0,0) ++(45:2) circle (2)}
\DeclareMathOperator{\C}{C}
\DeclareMathOperator{\util}{u}
%\setbeamersize{text margin left=1.5em,text margin right=1.5em} 
%\setbeamersize{text margin left=1.2cm,text margin right=1.2cm} 
\setbeamersize{text margin left=1.5em,text margin right=1.5em} 
%\usepackage{xr}
%\externaldocument{Econometrie1_UGA_P2e}
  \usepackage{eso-pic}
%\newcommand\AtPagemyUpperLeft[1]{\AtPageLowerLeft{%
%\put(\LenToUnit{0.9\paperwidth},\LenToUnit{0.85\paperheight}){#1}}}
%\AddToShipoutPictureFG{
 % \AtPagemyUpperLeft{{\includegraphics[width=1.1cm,keepaspectratio]{logoUGA2020.pdf}}}
%}%

%\setbeamercolor{title}{fg=black}
%\setbeamercolor{frametitle}{fg=black}
%\setbeamercolor{section in head/foot}{fg=black}
%\setbeamercolor{author in head/foot}{bg=Brown}
%\setbeamercolor{date in head/foot}{fg=Brown}
\AtBeginSection[]
  {
    \ifnum \value{framenumber}>1
      \begin{frame}<beamer>
      \frametitle{Outline}
      \tableofcontents[currentsection]
      \end{frame}
    \else
    \fi
  }
\setbeamertemplate{section page}
{
    \begin{centering}
    \begin{beamercolorbox}[sep=11pt,center]{part title}
    \usebeamerfont{section title}\thesection.~\insertsection\par
    \end{beamercolorbox}
    \end{centering}
}

%\titlegraphic{\includegraphics[width=1cm]{logoUGA2020.pdf}}
%\titlegraphic{\includegraphics[width=1cm]{logoUGA2020.pdf}}
\title[]{ \textbf{ÉCONOMIE INDUSTRIELLE}\footnote{Responsable du cours: Alexis Garapin.}\\(\textbf{UGA, L3 E2AD, S2})}
\subtitle{TRAVAUX DIRIGÉS: TD 3 \\   JEUX RÉPÉTÉS ET COLLUSION}
\date{\today}
\author{Michal W. Urdanivia\inst{*}}
\institute{\inst{*}UGA, Facult\'e d'\'Economie, GAEL, \\
e-mail:
 \href{
     mailto:michal.wong-urdanivia@univ-grenoble-alpes.fr}{michal.wong-urdanivia@univ-grenoble-alpes.fr}}

%\titlegraphic{\includegraphics[width=1cm]{logoUGA2020.pdf}
%}

\begin{document}

%%% TIKZ STUFF
\usetikzlibrary{positioning}
\usetikzlibrary{snakes}
\usetikzlibrary{calc}
\usetikzlibrary{arrows}
\usetikzlibrary{decorations.markings}
\usetikzlibrary{shapes.misc}
\usetikzlibrary{matrix,shapes,arrows,fit,tikzmark}
\usetikzlibrary{matrix,chains,positioning,decorations.pathreplacing,arrows}
\usetikzlibrary{shapes}
\usetikzlibrary{shapes.geometric, arrows}
\tikzset{   
        every picture/.style={remember picture,baseline},
        every node/.style={anchor=base,align=center,outer sep=1.5pt},
        every path/.style={thick},
        }
\newcommand\marktopleft[1]{
    \tikz[overlay,remember picture] 
        \node (marker-#1-a) at (-.3em,.3em) {};%
}
\newcommand\markbottomright[2]{%
    \tikz[overlay,remember picture] 
        \node (marker-#1-b) at (0em,0em) {};%
}
\tikzstyle{every picture}+=[remember picture] 
\tikzstyle{mybox} =[draw=black, very thick, rectangle, inner sep=10pt, inner ysep=20pt]
\tikzstyle{fancytitle} =[draw=black,fill=red, text=white]
\tikzstyle{observed}=[draw,circle,fill=gray!50]



\begin{frame}
\titlepage
\end{frame}
\begin{frame}
 \tableofcontents
    \end{frame}
%\begin{frame}
%\frametitle{Contenu}
%\tableofcontents[pausesections, pausesubsections]
%\end{frame}

%\section{Qu'est-ce que l’économétrie ? A quoi (à qui) ça sert ?}
%\frame{\sectionpage}
%\begin{frame}
%  \tableofcontents  
%\end{frame}
\section{Exercice 1: ALPHA }
\frame{\sectionpage}
\begin{frame}
[allowframebreaks]{\insertsection}
\framesubtitle{Données de l'exercice \\}
\begin{itemize}
    \item Monope ALPHA avec la fonCTion de coût:
    \begin{align}
    CT(q) &= q^2,
    \label{eq1}
    \end{align}
   \item Un marché caractérisé par la demande inverse:
   \begin{align}
     p(q)=120-q,
     \label{eq2}
   \end{align}
\end{itemize}
    \end{frame}

    \begin{frame}
[allowframebreaks]{\insertsection}
\framesubtitle{Question 1\\}
\begin{itemize}
\item \textbf{\underline{Problème}}: calculer le profit pour une tarification linéaire.
 
\item \textbf{\underline{Solution}}:

\begin{enumerate}[$\cdot$]
\item Le choix optimal de la firme est donné par:

\begin{align}
q^* &\coloneqq \argmax_q \pi(q),
\label{eq3}
\end{align}
où  $\pi(\cdot)$ est la fonCTion de profit donnée par:
\begin{align*}
\pi(q) &:= p(q)q - CT(q) = 120q – 2q^2,
\end{align*}
et la solution de \eqref{eq3}  donne:
\begin{align*}
\frac{\partial \pi}{\partial q}(q^*) = 0 &\Leftrightarrow 120-4q^* = 0 \Leftrightarrow q^* = 4,
\end{align*}
ce qui implique que,
\begin{align*}
p^* \coloneqq p(q^*) = 90, \quad \pi^* \coloneqq \pi(q^*) = 1800.
\end{align*}

\end{enumerate}
\end{itemize}

 \end{frame}

 \begin{frame}
  [allowframebreaks]{\insertsection}
  \framesubtitle{Question 2\\}
  \begin{itemize}
    \item \textbf{\underline{Problème}}: calculer  profit pour une discrimination parfaite
    \item \textbf{\underline{Solution}}: 
    \begin{enumerate}[$\cdot$]
      \item Le choix optimal de la firme est donné par:
      \item Le monopole produira  au plus $\tilde{q}$ tel que,
      \begin{align*}
        p(\tilde{q}) = c^m(\tilde{q}) \eqqcolon \frac{\partial CT}{\partial q}(\tilde{q}) 
        \Leftrightarrow  120 - \tilde{q} = 2\tilde{q}  \Leftrightarrow \tilde{q} = 40,
      \end{align*}
      ce qui implique que,
      \begin{align*}
        \tilde{p} \coloneqq p(\tilde{q}) = 80, \quad \tilde{\pi} \coloneqq \pi(\tilde{q}) = 1800.
        \end{align*}
      \item Dans ce cas le monopole va s’accaparer le surplus des consommateurs, que l’on peut calculer tel que:
       \begin{align*}
        %SC(\tilde{q}) &\coloneqq\int_0^{\tilde{q}}\left({p(q) - \tilde{p}}\right)\diff q =- \frac{\tilde{q}^{2}}{2} + 40\tilde{q} =40\frac{(120 - 80)}{2} = 800,
        SC(\tilde{q}) &\coloneqq\int_0^{\tilde{q}}\left({p(q) - \tilde{p}}\right)\diff q =40\tilde{q} - \frac{\tilde{q}^{2}}{2}  = 800,
       \end{align*}
       et le profit de la firme en discrimination parfaite est donc:
       \begin{align*}
        \tilde{\pi} &= SC(\tilde{q}) + \tilde{p}\tilde{q} - CT(\tilde{q}) = 2400.
       \end{align*}
       
    \end{enumerate}
  \end{itemize}

\end{frame}

\section{Exercice 2: THETA }
\frame{\sectionpage}
\begin{frame}
[allowframebreaks]{\insertsection}
\framesubtitle{Données de l'exercice \\}
\begin{itemize}
    \item Monope THETA avec la fonCTion de coût:
    \begin{align}
    CT(q) &= 20q \Rightarrow c^m(q) \coloneqq \frac{\partial CT}{\partial q}(q) = 20,
    \label{eq4}
    \end{align}
   \item Un marché caractérisé par la demande inverse:
   \begin{align}
     p(q)=50-q,
     \label{eq5}
   \end{align}
\end{itemize}
    \end{frame}
    
     \begin{frame}
  [allowframebreaks]{\insertsection}
  \framesubtitle{Question 1\\}
  \begin{itemize}
\item \textbf{\underline{Problème}}:  prix, quantité offerte et profit avec tarification linéaire.
    \item \textbf{\underline{Solution}}: 
  \begin{enumerate}[$\cdot$]
    \item THETA choisit la quantité optimale $q^*$ comme solution de:
    \begin{align*}
      q^* &= \argmax_q \pi(q),
    \end{align*}
    avec:
    \begin{align*}
      \pi(q) &\coloneqq p(q)q - CT(q) = \underbrace{RT(q)}_{\coloneqq p(q)q} - CT(q),
    \end{align*}
    et $q^*$ est définie par la c.p.o.,
    \begin{align}
      \frac{\partial \pi}{\partial q}(q^*) = 0 \Leftrightarrow \underbrace{R^m(q^*)}_{\coloneqq \frac{\partial RT}{\partial q}(q^*)} 
      = c^m(q^*),
      \label{eq6}
    \end{align}
    où  $R^m(\cdot)$ est la recette marginale. 
    \medskip
    Nous avons ici avec \eqref{eq4} et \eqref{eq5}, $q^*$ défini par,
    \begin{align*}
      \underbrace{50-2q^*}_{R^m(q^*)} = \underbrace{20}_{c^m(q^*)} \Leftrightarrow q^* = 15.
    \end{align*}
    et on calcule aussi:
    \begin{align*}
      p^* \coloneqq p(q^*) = 50-q^* = 35 &, \quad \pi^* \coloneqq\pi(q^*) = 225.
    \end{align*}
  \end{enumerate}
  \end{itemize}
    \end{frame}
    
    \begin{frame}
      [allowframebreaks]{\insertsection}
      \framesubtitle{Question 2\\}
      \begin{itemize}
    \item \textbf{\underline{Problème}}:  prix, quantité offerte et profit en discrimination parfaite.
        \item \textbf{\underline{Solution}}: 
      \begin{enumerate}[$\cdot$]
       \item En discrimination parfaite, le monopole peut tarifer
        chaque consommateur à son consentement à payer pour s’accaparer l’ensemble du surplus. 
        \item Il va donc produire jusqu’à ce que le consentement à payer soit plus faible que son coût de production 
        ce qui arrive pour $\tilde{q}$ tel que:
        \begin{align*}
          p(\tilde{q}) = c^m(\tilde{q}) &\Leftrightarrow \tilde{q} = 30.
        \end{align*}
        \item Son profit est ici égal au surplus des consommateurs:
        \begin{align*}
          %SC(\tilde{q}) &\coloneqq\int_0^{\tilde{q}}\left({p(q) - \tilde{p}}\right)\diff q =- \frac{\tilde{q}^{2}}{2} + 40\tilde{q} =40\frac{(120 - 80)}{2}
          \pi(\tilde{q}) &= SC(\tilde{q}) \coloneqq\int_0^{\tilde{q}}\left({p(q) - \tilde{p}}\right)\diff q =30\tilde{q} - \frac{\tilde{q}^{2}}{2}  = 450,
         \end{align*}
      \end{enumerate}
    \end{itemize}
      \end{frame}

      \begin{frame}
        [allowframebreaks]{\insertsection}
        \framesubtitle{Question 3\\}
        \begin{itemize}
          \item \textbf{\underline{Problème}}:  prix, quantité offerte et profit avec  tarification binôme
           (c.à.d., une partie fixe, une partie variable).

    \item \textbf{\underline{Solution}}: 
  \begin{enumerate}[$\cdot$]
   \item En tarification binôme, le monopole va pouvoir s’accaparer l’ensemble du surplus grâce à sa partie fixe.
   \item La partie variable de sa tarification est déterminée par le point/quantité ou il est juste encore rentable de 
   produire ce qui arrive d'après la question précédente pour une quantité $\bar{q} = 30$ avec un prix $\bar{p}\coloneqq p(\bar{q}) = 20$ 
   qui donne la partie variable de la tarification.
   \item La partie fixe $t$ est égale au surplus des consommateur pour un prix $\bar{p} = 20$, et d'après la question précédente il s'agit de $t = 450$. 
   \item On peut alors calculer le profit de la firme qui sera égal à 450.
  \end{enumerate}
\end{itemize}
  \end{frame}
   
  \section{Exercice 3: BETA }
\frame{\sectionpage}
\begin{frame}
[allowframebreaks]{\insertsection}
\framesubtitle{Données de l'exercice \\}
\begin{itemize}
    \item Monopole BETA vendant sur deux marches caractérisés par les demandes:
    \begin{align}
   p_1(q_1) & = 200-q_1,\\
   p_2(q_2) & = 300-q_2
    \label{eq7}
    \end{align}
   \item Les coûts de BETA sont donnés par:
   \begin{align}
     CT(q_1, q_2)&=(q_1+q_2)^2,
     \label{eq8}
   \end{align}
\end{itemize}
    \end{frame}

    \begin{frame}
      [allowframebreaks]{\insertsection}
      \framesubtitle{Question 1\\}
      \begin{itemize}
    \item \textbf{\underline{Problème}}:  quantités offertes.
    \item \textbf{\underline{Solution}}: 
    \begin{enumerate}[$\cdot$]
     \item Le profit de la firme est donné par:
     \begin{align}
      \pi(q_1, q_2) &\coloneqq p_1(q_1)q_1 +  p_2(q_2)q_2 - c(q_1, q_2),
      \label{eq9}
     \end{align}
     qu'elle maximise par rapport à $q_1$ et $q_2$ pour déterminer les quantités optimales $(q_1^*, q_2^*)$:
     \begin{align}
      (q_1^*, q_2^*) &=\argmax_{q_1, q_2}\pi(q_1, q_2).  
      \label{eq10}
     \end{align}
     \item La c.p.o. devant être satisfaite en $(q_1^*, q_2^*)$ correspond à un système de deux équations(annulation du gradient
     associé à $\pi(\cdot)$) à deux inconnues:
     \begin{align}
      \left\{
      \begin{array}{ll}
        \frac{\partial \pi}{\partial q_1}(q_1^*, q_2^*) &= 0\\
        \frac{\partial \pi}{\partial q_2}(q_1^*, q_2^*) &= 0
      \end{array}
      \right.
      \label{eq11}
     \end{align}
    \end{enumerate}
  \end{itemize}

    \end{frame}
\end{document}